%\section{Ionic diffusion in cementarious media}

\begin{frame}
%
\begin{block}{A physical situation}
\begin{itemize}%[<+->]
\item<+-> Reinforced concrete structures are exposed to chloride $Cl^-$ ;
\item<+-> $Cl^-$ penetrates the concrete, via its pores ;
\item<+-> It corrodes steel and makes the structure fragile.
\end{itemize}
\end{block}
%
\begin{block}{A model for electrodiffusion}<+->
\begin{description}
\item<+-> [The Nernst-Planck equation] $\dfrac{\partial{c_k}}{\partial{t}}=div\left(-D_k\left(grad(c_k)+\dfrac{F}{RT}grad(\Psi)\right)\right)$
\item<+-> [Poisson equation] $\epsilon_p \Delta \Psi=-\rho$
\end{description}
\end{block}
%
%\begin{block}{A computational issue}<+->
%\begin{itemize}
%\item<+-> A cimenteous material is made of numerous, and very variable sub-structures ;
%\item<+-> Above equations should be resolved on a huge spatial domain.
%\end{itemize}
%\end{block}
%
\end{frame}

\begin{frame}{Two scales of microstructure}
%
%\begin{block}{A two-scaled spatial model}<+->
%\begin{itemize}
%\item<+-> $\mathbf{x}$ is a macroscopical variable, namely the cell's position in the periodical structure ;
%\item<+-> $\mathbf{y}$ or $(y_1,y_2,y_3)$ is a microscopical variable, of spatial coordinates in a single cell ;
%\item<+-> $\mathbf{x}$ and $\mathbf{y}$ can be assumed to be independent.
%\end{itemize}
%\end{block}
%
\begin{block}{Two variables for space}<+->
%
\includegraphics[width=0.8\linewidth, height=3cm]{diff_ion/macro_micro_xvsy.png}

\par
\cite{th_KhaledB}, chapter 2.
%
%\pause
\begin{itemize}
\item<+-> $\mathbf{x}$ is the position in the left cylinder ;
\item<+-> $\mathbf{y}$ is the position in the right, square projection.
\end{itemize}
%
\end{block}
%
\end{frame}

\begin{frame}{Periodic homogenization}
%
\begin{block}{Hypotheses}<+->
\begin{itemize}%[<+->]
\item<+-> The porous medium can be modelized as periodical structure, with a small enough elementaty cell ;
\item<+-> The elementary cell summarize the medium macroscopical properties ;
\item<+-> $\mathbf{x}$ is the localization of a cell, $\mathbf{y}$ the position in a cell.
\end{itemize}
%
%\pause
\visible<2->{%
%\begin{figure}[H]
\begin{center}
\begin{tabular}{|c|c|c|c|}
\hline
%\subfloat[Single sphere]{
\includegraphics[width=0.2\linewidth,height=2cm]{../Figures3D/macro_micro_Lsurl1unique_par505035.png}%}
&
%\subfloat[Single cylinder]{
\includegraphics[width=0.2\linewidth,height=2cm]{../Figures3D/macro_micro_Lsurl1unique_cyl25.png}%}
&
%\subfloat[Two speres]{
\includegraphics[width=0.2\linewidth,height=2cm]{../Figures3D/macro_micro_Lsurl1diag_par505035.png}%}
&
%\subfloat[One sphere and one cylinder]{
\includegraphics[width=0.2\linewidth,height=2cm]{../Figures3D/macro_micro_Lsurl1sph_cyl_par505035.png}%}
\\
\hline
%\subfloat[Porosity : $0,82040562$]{
\includegraphics[width=0.2\linewidth,height=2cm]{../Figures3D/macro_micro_Lsurl5unique_par505035.png}%}
&
%\subfloat[Porosity : $0,803650459$]{
\includegraphics[width=0.2\linewidth,height=2cm]{../Figures3D/macro_micro_Lsurl5unique_cyl25.png}%}
&
%\subfloat[Porosity : $0,786895298$]{
\includegraphics[width=0.2\linewidth,height=2cm]{../Figures3D/macro_micro_Lsurl5diag_par505035.png}%}
&
%\subfloat[Porosity : $0,624056079$]{
\includegraphics[width=0.2\linewidth,height=2cm]{../Figures3D/macro_micro_Lsurl5sph_cyl_par505035.png}%}
\\
\hline
\end{tabular}
\end{center}
%\caption{Three-dimensional case : $5\times 5-$cells microstructure}
%\end{figure}
}
%
\end{block}
\end{frame}

\begin{frame}{The homogenized diffusion tensor}
%
\begin{block}{Homogenized diffusion-migration equation}<+->
\[\derp{c_k}{t}-div_{\mathbf{x}}\left(D_k^{hom}\derp{c_k}{\mathbf{x}}+D_k^{hom}\frac{F}{RT}z_k \derp{\Psi}{\mathbf{x}}\right)=0\]
\end{block}
%
\begin{block}{$D_k^{hom}$ and the microstructure}<+->% : the $\chi$ vector field}<+->
\begin{itemize}
\item<+-> \[D_k^{hom}:=\int\limits_{\Omega_f}\left(I+\derp{\chi}{\mathbf{y}}^T\right)\text{d}\Omega\]
\item<+-> We will focus on this problem by now.
\end{itemize}
\end{block}
%
\end{frame}

\begin{frame}{A parametrized vector field}
%
\begin{block}{Vector $\chi$ : strong problem}<+->
\begin{itemize}
\item<+-> For avery $\rho$, $\chi(\mathbf{y},\rho)$ satisfies :
\[%
\left\{%
\begin{array}{lccr}
Div_y \left( \dfrac{\partial{\chi}}{\partial{y}}^T\right) &=& 0&\text{ on }\Omega_f \\
\dfrac{\partial{\chi}}{\partial{y}} \cdot n_{sf}&=&-n_{sf}&\text{ on }\Gamma_{sf}
\end{array}
\right.
\]
\item<+-> %General solution $\chi(\mathbf{y},\rho)$, 
$\rho$ can be the porosity ;
\item<+-> $\Omega_f$ and $\Gamma{sf}$ depend on $\rho$.
\end{itemize}
\end{block}
%
\end{frame}

\begin{frame}{Weak formulation}% of the problem}
%
\begin{block}{}
\[%
\forall v \in V , %
\int\limits_{\Omega_f}Tr\left(\left(grad_y\chi\right)^T\cdot grad_y v\right)\text{d}\Omega=%
-\int\limits_{\Gamma{sf}}n_{sf}\cdot v\text{d}s%
\]
\end{block}

\pause
\begin{itemize}
\item<+-> This can be done with Finite Element Method ;
\item<+-> We see that POD-ROM gives better results.
\end{itemize}
%
\end{frame}