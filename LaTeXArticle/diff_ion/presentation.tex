\section{Ionic diffusion in cementarious media}

\begin{frame}
%
\begin{block}{A physical situation}
\begin{itemize}%[<+->]
\item<+-> Reinforced concrete structures are exposed to chloride $Cl^-$
\item<+-> $Cl^-$ penetrates the concrete, via its pores
\item<+-> It corrodes steel and makes the structure fragile.
\end{itemize}
\end{block}
%
\begin{block}{A model for electrodiffusion}<+->
\begin{description}
\item<+-> [The Nernst-Planck equation] $\dfrac{\partial{c_k}}{\partial{t}}=div\left(-D_k\left(grad(c_k)+\dfrac{F}{RT}grad(\Psi)c_k\right)\right)$
\item<+-> [Poisson equation] $\epsilon_v \Delta \Psi=-\rho$
\item<+-> [] $\displaystyle \rho=\sum\limits_k z_k c_k$, $z_k$ is the valence of ion $k$.
\end{description}
\end{block}
%
\end{frame}

\begin{frame}{Two scales of microstructure}
%
\begin{block}{Two variables for space}<+->
%
\includegraphics[width=0.8\linewidth, height=3cm]{../Figures2D/macro_micro_xvsy.png}

\par
Bourbatache, PhD thesis, chapter 2.
%
%\pause
\begin{itemize}
\item<+-> $\mathbf{x}$ is the variable at macroscale
\item<+-> $\mathbf{y}$ is the variable at microscale
\end{itemize}
%
\end{block}
%
\end{frame}

\begin{frame}%{Periodic homogenization}
%
\begin{block}{Periodic homogenization : macroscopic scale}<+->
We modelize the porous medium as a periodical structure

\vspace{-0.2cm}
\begin{center}
\begin{tabular}{|c|c|c|c|}
\hline
%\subfloat[Porosity : $0,82040562$]{
%\visible<2->{
\includegraphics[width=0.2\linewidth,height=2cm]{../Figures3D/macro_micro_Lsurl5unique_par505035.png}%}%}
&
%\subfloat[Porosity : $0,803650459$]{
%\visible<2->{
\includegraphics[width=0.2\linewidth,height=2cm]{../Figures3D/macro_micro_Lsurl5unique_cyl25.png}%}%}
&
%\subfloat[Porosity : $0,786895298$]{
%\visible<2->{
\includegraphics[width=0.2\linewidth,height=2cm]{../Figures3D/macro_micro_Lsurl5diag_par505035.png}%}%}
&
%\subfloat[Porosity : $0,624056079$]{
%\visible<2->{
\includegraphics[width=0.2\linewidth,height=2cm]{../Figures3D/macro_micro_Lsurl5sph_cyl_par505035.png}%}%}
\\
\hline
\end{tabular}
\end{center}
\vspace{-0.2cm}

Variable $\mathbf{x}$ : position of each voxel
%
\end{block}
%
\begin{block}{Periodic homogenization : microscopic scale}<+->
One voxel = an elementary cell, summarizing the structure's properties

\vspace{-0.2cm}
\begin{center}
\begin{tabular}{|c|c|c|c|}
\hline
%\subfloat[Single sphere]{
%\visible<3->{
\includegraphics[width=0.2\linewidth,height=2cm]{../Figures3D/macro_micro_Lsurl1unique_par505035.png}%}%}
&
%\subfloat[Single cylinder]{
%\visible<3->{
\includegraphics[width=0.2\linewidth,height=2cm]{../Figures3D/macro_micro_Lsurl1unique_cyl25.png}%}%}
&
%\subfloat[Two speres]{
%\visible<3->{
\includegraphics[width=0.2\linewidth,height=2cm]{../Figures3D/macro_micro_Lsurl1diag_par505035.png}%}%}
&
%\subfloat[One sphere and one cylinder]{
%\visible<3->{
\includegraphics[width=0.2\linewidth,height=2cm]{../Figures3D/macro_micro_Lsurl1sph_cyl_par505035.png}%}%}
\\
\hline
\end{tabular}
\end{center}
\vspace{-0.2cm}

Variable $\mathbf{y}$ : position in the elementary cell.
%
\end{block}
\end{frame}

\begin{frame}{The homogenized diffusion tensor}
%
\begin{block}{Homogenized diffusion-migration equation}<+->
\[\derp{c_k}{t}-div_{\mathbf{x}}\left(D_k^{hom}\derp{c_k}{\mathbf{x}}+D_k^{hom}\frac{F}{RT}z_k \derp{\Psi}{\mathbf{x}}\right)=0\]
\end{block}
%c_k 
\begin{block}{$D_k^{hom}$ and the microstructure}<+->% : the $\chi$ vector field}<+->
%\item<+-> 
\[D_k^{hom}:=\int\limits_{\Omega_f}\left(I+\derp{\chi}{\mathbf{y}}^T\right)\text{d}\Omega\]
\end{block}
%
\begin{block}{Vector $\chi$ : local problem}<+->
\vspace{-0.3cm}
%\begin{itemize}
%\item<+-> %
\[%
\left\{%
\begin{array}{llccr}
\text{Fluid domain :}&Div_y \left( \dfrac{\partial{\chi}}{\partial{y}}^T\right) &=& 0&\text{ on }\Omega_f \\
\text{Solid-fluid interface :}&\dfrac{\partial{\chi}}{\partial{y}} \cdot n_{sf}&=&-n_{sf}&\text{ on }\Gamma_{sf}
\end{array}
\right.
\]

%\item<+-> 
%\pause
We focus on this problem in this work.
%\end{itemize}
\end{block}
%
\end{frame}

\begin{frame}%{Microstructure variability}%transition : rom
%
\begin{block}{Microstructure parameters}<+->
%
\begin{multicols}{4}
A set of real numbers $\rho_1, \rho_2 \dots \rho_{N_{par}}$

\columnbreak
\visible<2->{For example : porosity, average size of inclusions.}

\columnbreak

%\pause
\visible<3->{%
\includegraphics[width=\linewidth, height=2.5cm]{../Figures2D/Npar4Cr.png}%par4crop500ffT30000.png}%

%\footnotesize{A. Moreau 2018}
}

\columnbreak

%\pause
\visible<3->{%
$N_{par}=4$ : %
%\vspace{-0.5cm}

\par
mean and variance for yellow or grey inclusions.
}
\end{multicols}
%
\end{block}
%
\begin{block}{Our aim : computing $D^{hom}(\rho_1,\rho_2,\dots ,\rho_{N_{par}})$}<4->
%
\begin{tabular}{|c|c|c|}
\hline
\visible<4->{\includegraphics[width=0.27\linewidth, height=1.9cm]{../Figures2D/meb_cem_1.png}}&%
\visible<5->{\includegraphics[width=0.27\linewidth, height=1.9cm]{../Figures2D/meb_cem_2.png}}&%
\visible<7->{\includegraphics[width=0.27\linewidth, height=1.9cm]{../Figures2D/meb_cem_3.png}}%
\\
\hline
\visible<4->{$(\rho_1^1 ,\rho_2^1, \dots ,\rho_{N_{par}}^1)$}&%
\visible<5->{$(\rho_1^2 ,\rho_2^2, \dots ,\rho_{N_{par}}^2)$}&%
\visible<7->{$(\rho_1^{new} ,\rho_2^{new}, \dots ,\rho_{N_{par}}^{new})$}%
\\
\hline
\end{tabular}

\begin{description}
\item<6-> [Training] Learn $D^{hom}$ on $(\rho_1^1 ,\rho_2^1, \dots ,\rho_{N_{par}}^1)$ and $(\rho_1^2 ,\rho_2^2, \dots ,\rho_{N_{par}}^2)$% (expensive)
\item<7-> [Reduced Order Model] Derive quickly $D^{hom}(\rho_1^{new} ,\rho_2^{new}, \dots ,\rho_{N_{par}}^{new})$.
\end{description}
%
%\par
\footnotesize{Images : Chalen\c con et al., 2009}
\end{block}
%
\end{frame}