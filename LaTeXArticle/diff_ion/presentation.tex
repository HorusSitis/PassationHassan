%\section{Ionic diffusion in cementarious media}

\begin{frame}
%
\begin{block}{A physical situation}
\begin{itemize}%[<+->]
\item<+-> Reinforced concrete structures are exposed to chloride $Cl^-$ ;
\item<+-> $Cl^-$ penetrates the concrete, via its pores ;
\item<+-> It corrodes steel and makes the structure fragile.
\end{itemize}
\end{block}
%
\begin{block}{A model for electrodiffusion}<+->
\begin{description}
\item<+-> [The Nernst-Planck equation] $\dfrac{\partial{c_k}}{\partial{t}}=div\left(-D_k\left(grad(c_k)+\dfrac{F}{RT}grad(\Psi)c_k\right)\right)$
\item<+-> [Poisson equation] $\epsilon_v \Delta \Psi=-\rho$
\end{description}
\end{block}
%
\end{frame}

\begin{frame}{Two scales of microstructure}
%
\begin{block}{Two variables for space}<+->
%
\includegraphics[width=0.8\linewidth, height=3cm]{../Figures2D/macro_micro_xvsy.png}

\par
Bourbatache, PhD thesis, chapter 2.
%
%\pause
\begin{itemize}
\item<+-> $\mathbf{x}$ is the variable at macroscale ;
\item<+-> $\mathbf{y}$ is the variable at microscale.
\end{itemize}
%
\end{block}
%
\end{frame}

\begin{frame}{Periodic homogenization}
%
\begin{block}{Hypotheses}<+->
\begin{itemize}%[<+->]
\item<+-> The porous medium can be modelized as periodical structure, with a small enough elementaty cell ;
\item<+-> The elementary cell summarizes the medium macroscopical properties ;
\item<+-> $\mathbf{x}$ is the localization of a cell, $\mathbf{y}$ the position in a cell.
\end{itemize}
%
%\pause
\visible<2->{%
%\begin{figure}[H]
\begin{center}
\begin{tabular}{|c|c|c|c|}
\hline
%\subfloat[Single sphere]{
\includegraphics[width=0.2\linewidth,height=2cm]{../Figures3D/macro_micro_Lsurl1unique_par505035.png}%}
&
%\subfloat[Single cylinder]{
\includegraphics[width=0.2\linewidth,height=2cm]{../Figures3D/macro_micro_Lsurl1unique_cyl25.png}%}
&
%\subfloat[Two speres]{
\includegraphics[width=0.2\linewidth,height=2cm]{../Figures3D/macro_micro_Lsurl1diag_par505035.png}%}
&
%\subfloat[One sphere and one cylinder]{
\includegraphics[width=0.2\linewidth,height=2cm]{../Figures3D/macro_micro_Lsurl1sph_cyl_par505035.png}%}
\\
\hline
%\subfloat[Porosity : $0,82040562$]{
\includegraphics[width=0.2\linewidth,height=2cm]{../Figures3D/macro_micro_Lsurl5unique_par505035.png}%}
&
%\subfloat[Porosity : $0,803650459$]{
\includegraphics[width=0.2\linewidth,height=2cm]{../Figures3D/macro_micro_Lsurl5unique_cyl25.png}%}
&
%\subfloat[Porosity : $0,786895298$]{
\includegraphics[width=0.2\linewidth,height=2cm]{../Figures3D/macro_micro_Lsurl5diag_par505035.png}%}
&
%\subfloat[Porosity : $0,624056079$]{
\includegraphics[width=0.2\linewidth,height=2cm]{../Figures3D/macro_micro_Lsurl5sph_cyl_par505035.png}%}
\\
\hline
\end{tabular}
\end{center}
%\caption{Three-dimensional case : $5\times 5-$cells microstructure}
%\end{figure}
}
%
\end{block}
\end{frame}

\begin{frame}{The homogenized diffusion tensor}
%
\begin{block}{Homogenized diffusion-migration equation}<+->
\[\derp{c_k}{t}-div_{\mathbf{x}}\left(D_k^{hom}\derp{c_k}{\mathbf{x}}+D_k^{hom}\frac{F}{RT}z_k \derp{\Psi}{\mathbf{x}}\right)=0\]
\end{block}
%
\begin{block}{$D_k^{hom}$ and the microstructure}<+->% : the $\chi$ vector field}<+->
%\item<+-> 
\[D_k^{hom}:=\int\limits_{\Omega_f}\left(I+\derp{\chi}{\mathbf{y}}^T\right)\text{d}\Omega\]
\end{block}
%
\begin{block}{Vector $\chi$ : local problem}<+->
\vspace{-0.3cm}
%\begin{itemize}
%\item<+-> %
\[%
\left\{%
\begin{array}{llccr}
\text{Fluid domain :}&Div_y \left( \dfrac{\partial{\chi}}{\partial{y}}^T\right) &=& 0&\text{ on }\Omega_f \\
\text{Solid-fluid interface :}&\dfrac{\partial{\chi}}{\partial{y}} \cdot n_{sf}&=&-n_{sf}&\text{ on }\Gamma_{sf}
\end{array}
\right.
\]

%\item<+-> 
%\pause
We focus on this problem in this work.
%\end{itemize}
\end{block}
%
\end{frame}

\begin{frame}%{Microstructure variability}%transition : rom
%
\begin{block}{Microstructure parameters}<+->
%
%\begin{itemize}
\begin{multicols}{3}
%\item<+-> 
$\rho^1, \rho^2 \dots$ can be porosity, average size of inclusions etc.
%\item<+-> %

\columnbreak

%\pause
\visible<+->{%
A structure with 4 parameters : %
%\vspace{-0.5cm}

\par
mean and variance for two sorts of inclusions.}

\columnbreak

%\pause
\visible<+->{%
\includegraphics[width=\linewidth, height=2.5cm]{../Figures2D/par4crop500ffT30000.png}%
}
\end{multicols}
%
%\end{itemize}
%
\end{block}
%
\begin{block}{The expensive computing of $D^{hom}(\rho^1,\dots ,\rho^N)$}<+->
%
\begin{tabular}{|c|c|c|}
\hline
\includegraphics[width=0.27\linewidth, height=1.9cm]{../Figures2D/meb_cem_1.png}&%
\includegraphics[width=0.27\linewidth, height=1.9cm]{../Figures2D/meb_cem_2.png}&%
\includegraphics[width=0.27\linewidth, height=1.9cm]{../Figures2D/meb_cem_3.png}%
\\
\hline
$(\hat{\rho^1},\dots ,\hat{\rho^N})$&%
$(\tilde{\rho^1},\dots ,\tilde{\rho^N})$&%
$(\overline{\rho^1},\dots ,\overline{\rho^N})$%
\\
\hline
\end{tabular}

\begin{description}
\item<+-> [Training] Learn $D^{hom}$ on $(\hat{\rho^1},\dots ,\hat{\rho^N})$ and $(\tilde{\rho^1},\dots ,\tilde{\rho^N})$
\item<+-> [ROM] Quickly derive $D^{hom}(\overline{\rho^1},\dots ,\overline{\rho^N})$.
\end{description}
%
\par
Images : Chalen\c con et al., 2009
\end{block}
%
\end{frame}