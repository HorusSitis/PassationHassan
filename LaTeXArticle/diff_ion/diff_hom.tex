%\section{Ionic diffusion in porous media}
In this section, we present the method of periodic homogenization for the diffusion of ionic species in cementitious material.

\par
%%% à mettre dans le résumé éventuellement
Indeed, the life service of steel concrete is affected by the diffusion of ionic species, %
like chloride for constructions on the coasts or off-shore. %
Chloride penetrate the cementitious material, which is porous, and then deteriorates the steel structure, making the concrete fragile.

\subsection{Ionic species in saturated porous media : The Nernst-Planck-Poisson equation}

We consider a saturated porous medium $\Omega$, which consists into an solid phase $\Omega_s$ and a fluid phase $\Omega_f$.

\par
The following equations describe the dynamics, in the fluid phase $\Omega_f$, of the respective concentrations : $c_k$ , of a family of ionic species with index $k$ ; %
$z_k$ is the valence of ion $k$.

\begin{equation}\label{npp_f}
\left\{%
\begin{array}{lcl}
\dfrac{\partial{c_k}}{\partial{t}}&=&div\left(-D_k\left(grad(c_k)+\dfrac{F}{RT}grad(\Psi)\right)\right)\\
\epsilon_p \Delta \Psi&=&-\rho
\end{array}
\right.
\end{equation}

For the diffusion equations $D_k$ is the self-diffusion of ion $k$, $F$ and $R$ are the Faraday and perfect gas constants, $T$ the temperature ; %
for the Poisson equation $\Psi$ is the electrical potential, $\epsilon_v$ the dielectrical constant of the porous medium and $\rho$ the volumetric electrical charge.

\par
The above system is completed by the Neumann condition on $\Gamma_{sf}$ :

\begin{equation}\label{npp_sf}
\left\{%
\begin{array}{lcr}
D_k\left(grad(c_k)+\dfrac{F}{RT}grad(\Psi)\right)\cdot \vec{n}_{sf}&=&0\\
grad \Psi\cdot \vec{n}_{sf}&=&0
\end{array}
\right.
\end{equation}

In this model, we don't care about electrical interactions between the ions and the solid phase.

\subsection{Periodic homogenization}

The point is, these equations describe the dynamics of ionic concentrations in a medium with many heterogeneities. %
Homogenization methods attempt to replace these equations by another system depending on macroscopic variables, %
\og{} blurring \fg{} microscopical heterogeneities.

\par
More precisely \dots







%\[\derp[2]{\chi}{y}\text{ and }\derp{\chi}{y}\]




\subsection{The homogenized diffusion tensor}

%\subsubsection{}

The above method eventually gives the following equation, which depends only of the $x-$ spatial coordinates :








where $\chi(y)$ is a vector field, defined on the fluid phase of the elementary cell, satisfying the equations :

%\newcommand{}
%Grad \chi
%\dfrac{\partial{\chi}}{\partial{y}}
\begin{description}
\item[Equation on the pores] 
\begin{equation}
\left\{%
\begin{array}{lccr}
Div_y \left( \dfrac{\partial{\chi}}{\partial{y}}^T\right) &=& 0&\text{ on }\Omega_f \\
\dfrac{\partial{\chi}}{\partial{y}} \cdot n_{sf}&=&-n_{sf}&\text{ on }\Gamma_{sf}
\end{array}
\right.
%\chi\text{ in periodic}
\label{chistr}
\end{equation}
\item[Periodicity] $\chi$ is periodic on $\Gamma_{ff}$
\item[Mean value condition] $\langle\chi\rangle_{\Omega_f}=0$
\end{description}

\ligneinter
A proof of these statements can be found in \cite{th_khaledB}, chapter 2.

\begin{rema}%%%à écrire dans le résumé, ou un paragraphe de conclusion ?
In this section, we focused on Nernst-Plank equation, which describes the diffusion of ionic species, like chloride, in the fluid domain of cementitious material.

\par
However the periodic homogenization method can be applied to other mechanisms of degradation of concrete structures : %
see section 2 of \cite{WalMill16} for a homogenized equation of moisture transport in cementitious material.

\par
Everything that we do in sections \ref{rom} and \ref{res} stays true in these contexts.
\end{rema}
\begin{comment}
\subsubsection{Homogenized Richards Equation \cite{WalMill16}}

\begin{equation}\label{ric_wat}
\dfrac{\partial{\theta_l}}{\partial{t}}-div_x\left(\mathbf{D^{hom}_{\theta}}\dfrac{\partial{\theta_l}}{\partial{x}}\right)=0
\end{equation}

where $\theta_l$ is the water content : $\theta_l=\dfrac{\Omega_l}{\Omega}$ ; and

\[\mathbf{D^{hom}_{\theta}}=-\Lambda_{ll}\dfrac{\partial{P_c}}{\partial{\theta_l}}+\dfrac{1}{\rho_l}\mathbf{D^{hom}_{v}}\dfrac{\partial{\rho_v}}{\partial{\theta_l}}\]

denotes the homogenized diffusion tensor of moisture, $\rho_l$ and $\rho_v$ are liquid and vapour densities, respectively. $\Lambda_{ll}$ is the Darcy tensor of permeability, %
$P_c$ the macroscopic capillary pressure and $\mathbf{D^{hom}_{v}}$ the homogenized tensor of water vapour given by

\[\mathbf{D^{hom}_{v}}=\dfrac{1}{|\Omega|}\int\limits_{\Omega_{g}}D_v \left(I+\frac{\partial{\chi}}{\partial{y}}^{T}\right)\text{d}\Omega\]

where $D_v$ is the self-diffusion coefficient of water-vapour.

\par
The vector $\chi (y)$ is solution of the same system as for Nernst-Planck-Poisson equation, if we write $\Omega_g$ and $\Gamma_{gs}$ in place of $\Omega_f$ and $\Gamma_{sf}$.
\end{comment}

\subsection{Weak formulation of problem \ref{chistr}}

The weak formulation of the equation on $\Omega_{sf}$ is written :

\[\forall v \in V , \int\limits_{\Omega_{sf}}Div\left(\left(grad_y \chi\right)^{T}\right)\cdot v\text{d}\Omega=0\]

where $V$ is the space of test functions.

\par
Using the formula :

\[Div(A\cdot w)=Div(A)\cdot w+Tr(A\cdot w)\]

which is true for all $2-$tensor field $A$ et all vector field $w$, we obtain :

\[\forall v \in V , \int\limits_{\Omega_{sf}}Div\left(\left(grad_y \chi\right)^{T}\cdot v\right)\text{d}\Omega-%
\int\limits_{\Omega_f}Tr\left(\left(grad_y\chi\right)^T\cdot grad_y v\right)\text{d}\Omega=0\]

which gives, acccording to the divergence theorem :

\[\forall v \in V , \int\limits_{\Gamma_{sf}\cup\Gamma{ff}}\left(\left(grad_y \chi\right)^{T}\cdot v\right)\cdot n\text{d}s-%
\int\limits_{\Omega_f}Tr\left(\left(grad_y\chi\right)^T\cdot grad_y v\right)\text{d}\Omega=0\]

The Neumann condition in \ref{chistr} then implies :

\begin{equation}\label{chiw}
\forall v \in V , %
\int\limits_{\Omega_f}Tr\left(\left(grad_y\chi\right)^T\cdot grad_y v\right)\text{d}\Omega=%
-\int\limits_{\Gamma{sf}}n_{sf}\cdot v\text{d}s%
+\int\limits_{\Gamma{ff}}\left(\left(grad_y \chi\right)^{T}\cdot v\right)\cdot n_{ff}\text{d}s
\end{equation}

$\left(grad_y \chi\right)^{T}$ and $v$ are periodic, $n_{ff}$ is anti-periodic : %
in equation \ref{chiw} $\displaystyle\int\limits_{\Gamma{ff}}\left(\left(grad_y \chi\right)^{T}\cdot v\right)\cdot n_{ff}\text{d}s$ vanishes so we conclude :

\begin{equation}\label{chiwper}
\forall v \in V , %
\int\limits_{\Omega_f}Tr\left(\left(grad_y\chi\right)^T\cdot grad_y v\right)\text{d}\Omega=%
-\int\limits_{\Gamma{sf}}n_{sf}\cdot v\text{d}s
\end{equation}

which is, when combinated with the mean-value condition $\langle\chi\rangle_{\Omega_f}=0$, the weak formulation of \ref{chistr}.

\subsection{Solutions with FEniCS : examples of microstructures}

\subsubsection{The two-dimensional case}




%%%Maillages, raffinement : exemples, figures
\begin{figure}[H]
\begin{center}
\begin{tabular}{|c|c|c|c|}
\hline
\subfloat[A single fluid domain]{\includegraphics[width=0.24\linewidth,height=3.2cm]{../Figures2D/mesh_fixe.png}}
&%\hfill
\subfloat[A solid centered inclusion]{\includegraphics[width=0.24\linewidth,height=3.2cm]{../Figures2D/mesh_r_per505025.png}}
&%\hfill
\subfloat[Four inclusions the vertices]{\includegraphics[width=0.24\linewidth,height=3.2cm]{../Figures2D/mesh_r_per0025.png}}
&%\hfill
\subfloat[One inclusion on both sides]{\includegraphics[width=0.24\linewidth,height=3.2cm]{../Figures2D/mesh_r_per05025.png}}
\\
\hline
\end{tabular}
\end{center}
\caption{Examples of meshes performed with FEniCS}
\label{2d_mesh}
\end{figure}




\subsubsection{The three-dimensional case}

Algorithms are the same as in the two-dimensional case, but we will see that CPU times is much biggers, %
even if the mesh resolution is reasonable.





%%%Inclusion au centre
%%%r de 0.05 à 0.40
%%%Dhom_k : 0.93455015305 ; 0.93455015305 ; 0.93455015305 ; 0.937713361944 ; 0.942352510417 ; 0.94383619236 ; 0.941721377617






\subsection{Performances with Finite Element Method}

$\chi$, and thus the homogenized diffusion tensor $D_k^{hom}$, depends highly of the geometry of the microstructure of the the porous medium. %
Moreover, the CPU time for the resolution of the weak problem with Finite Element Method can be considerable, especially in the three-dimensional case, %
as the following table shows us.

%%%Table : temps CPU pour le calcul de \chi en 2-3D, préciser la résolution du maillage.


















