\section{Reduced Order Models}

\subsection{Definition}

\begin{frame}{Proper Orthogonal Decomposition}
%
\begin{block}{Motivation}<+->
%
\begin{itemize}
\item<+-> Compute a family of functions $w(\mathbf{x},\tau)$, depending of a parameter $\tau$
\item<+-> Each function $\mathbf{x}\mapsto w(\mathbf{x},\tau )$ lies in the same Hilbert space $H$.
%\item<+-> $w(\cdot ,\tau)$ lies in an Hilbert space $H$ for each $\tau$
\end{itemize}
%
\end{block}
%
\begin{block}{What we want}<+->
\begin{itemize}
\item<+-> $\displaystyle w(\mathbf{x},\tau)\simeq \sum\limits_{i=1}^{N_{rom}}a_i(\tau)\phi_i(\mathbf{x})$
\item<+-> $N_{rom}$ must be small.
\end{itemize}
\end{block}
%
\begin{block}{Properties of the spatial modes}<+->
\begin{description}
\item<+-> [Orthogonality] $(\phi_i)$ is an orthogonal family in $H$
\item<+-> [Optimality of $(\phi_i)$] %
$\phi_l= \arg\max\limits_{\phi\in H}\left\langle\left(\phi\left|\sum\limits_{k=1}^{l-1}\left(w|\phi_k\right)_H\phi_k\right.\right)_H\right\rangle$%
\end{description}
\end{block}
%
\end{frame}

\begin{frame}{Method of snapshots}
%
\begin{block}{}<+->
\begin{itemize}
\item<+-> We have a sample $w(\mathbf{x},\tau^j)_{j=1}^{N_{snap}}$ of spatial functions, called \emph{snapshots}
\item<+-> They come from experimentation or numerical simulation
\item<+-> We write : $\phi_i(\mathbf{x})=\sum\limits_{j=1}^{N_{snap}}A_i(\tau^j) w(\mathbf{x},\tau^j)$
\end{itemize}
\end{block}
%
\begin{block}{Computing $(A_i)$}<+->%The new POD problem}<+->%Fredholm equation}<+->
%
\begin{description}
\item<+-> [Eigenvalue problem] $C_{\tau,\tau'}A_i(\tau)=\lambda_i A_i(\tau)$
\item<+-> [Time-correlation tensor] $C(\tau,\tau ')=\dfrac{1}{N_{\tau}}\int\limits_{\Omega} w(\mathbf{x},\tau)w(\mathbf{x},\tau')\text{d}\mathbf{x}$
\item<+-> $C$ is a $N_{snap}\times N_{snap}-$ symmetric positive matrix.
\end{description}
%
\end{block}
%
\end{frame}

\begin{frame}{Reduced basis}
%
\begin{block}{}<+->
\begin{itemize}
\item<+-> We are given a POD basis $(\phi_i)_{i=1}^{N_{snap}}$ with a set $\lambda_i$ of associated eigenvalues ;
\item<+-> We choose the first $N_{rom}$ vectors of $\Phi$ so that : %
$\dfrac{Ener_{N_{rom}}}{Ener_{N_{snap}}}\text{ \textit{id.est} }\dfrac{\sum\limits_{k=1}^{N_{rom}}\lambda_k}{\sum\limits_{k=1}^{N_{snap}}\lambda_k} >99,9\dots \%$ ;
\item<+-> In practical cases, $N_{rom}$ is small.
\end{itemize}
\end{block}
%
\end{frame}

\begin{frame}{Reduced Order Model : for a parameter $\tau^{new}$}
\begin{block}{Galerkin projection}<+->
%
\begin{description}
\item<+-> [For a given problem] $\mathcal{A}^{\mathbf{x},\tau}\left(w(\mathbf{x},\tau)\right)=f(\mathbf{x},\tau)$
\item<+-> [ROM] $w(\mathbf{x},\tau^{new})\simeq\sum\limits_{i=1}^{N_{rom}} a_i(\tau^{new})\phi_i(\mathbf{x})$
\item<+-> [The problem in $a_k(\tau^{new})$ : projection on $\phi_i$] %
\[%
\begin{split}
\left\langle\left.\mathcal{A}\left(\sum\limits_{k=1}^{N_{rom}} a_k(\tau^{new})\phi_k(\mathbf{x})\right) \right|\phi_i\left(\mathbf{x}\right)\right\rangle_H =& %
\left\langle\left(\mathbf{x},\tau^{new}\right)|\phi_i(\mathbf{x})\right\rangle_H\\%
%
&+\left\langle\left.\mathcal{R}_{N_{rom}}(\mathbf{x},\tau^{new})\right|\phi_i\right\rangle_H
\end{split}
\]
\item<+-> [Remainder term] $\mathcal{R}_{N_{rom}}(\mathbf{x},\tau^{new})$ is orthogonal to $\phi_i$
\end{description}
%\visible<4->{
%\[%
%\left(\left.\mathcal{A}\left(\sum\limits_{k=1}^{N_{rom}} a_k(\tau^{new})\phi_k(\mathbf{x})\right) \right|\phi_i\left(\mathbf{x}\right)\right)_H = %
%\left(f\left(\mathbf{x},\tau^{new}\right)|\phi_i(\mathbf{x})\right)_H+\mathcal{R}_{N_{rom}}(\mathbf{x},\tau^{new})
%\]
%}
%
\end{block}
%
\end{frame}

\subsection{ROM for periodic homogenization}

\begin{frame}%{ROM for periodic homogenization}
%
\begin{block}{Step 1 : compute a sample $\chi(\mathbf{y},\mathbf{\rho}^j)$ of vector fields}%<+->
%
\only<1>{%
\[%
\left\{%
\begin{array}{llccr}
\text{Fluid domain :}&Div_y \left( \dfrac{\partial{\chi}}{\partial{y}}^T\right) &=& 0&\text{ on }\Omega_f(\rho^j) \\
\text{Solid-fluid interface :}&\dfrac{\partial{\chi}}{\partial{y}} \cdot n_{sf}&=&-n_{sf}&\text{ on }\Gamma_{sf}(\rho^j)
\end{array}
\right.
\]
}
\only<2->{%
Weak formulation :
\[%
\int\limits_{\Omega_f(\rho^j)}Tr\left(\left(grad_y\chi(\mathbf{y},\rho^j))\right)^T\cdot grad_y v(\mathbf{y})\right)\text{d}\Omega%
-\int\limits_{\Gamma{sf}(\rho^j)}n_{sf}\cdot v(\mathbf{y})\text{d}s=0%
\]
We use Finite Element Method

%\par
}
%
\begin{center}
\begin{tabular}{|c|c|}%c|}
\hline
\visible<1-2>{\includegraphics[width=0.4\linewidth, height=2.9cm]{../Figures2D/meb_cem_1.png}}%
&%
\visible<1-2>{\includegraphics[width=0.4\linewidth, height=2.9cm]{../Figures2D/meb_cem_2.png}}%
%&%
%\visible<1->{\includegraphics[width=0.27\linewidth, height=1.9cm]{../Figures2D/meb_cem_3.png}}%
\\
\hline
\visible<1-2>{$(\rho_1^1 ,\rho_2^1, \dots ,\rho_{N_{par}}^1)$}%
&%
\visible<1-2>{$(\rho_1^2 ,\rho_2^2, \dots ,\rho_{N_{par}}^2)$}%
%&%
%\visible<1->{$(\rho_1^{new} ,\rho_2^{new}, \dots ,\rho_{N_{par}}^{new})$}%
\\
\hline
\end{tabular}
\end{center}

%\smallskip
\only<1-2>{\footnotesize{for each multi-parameter $(\rho_1^j,\dots ,\rho_{N_{par}}^j)$, $j=1 , 2\dots$ of the training sample.}}
%\only<3->{}
%
\end{block}
%
\end{frame}

\begin{frame}%{POD}
%
\begin{block}{Step 2 : a common function space for snapshots}
\begin{itemize}
\item<+-> The solutions $\chi(\mathbf{y},\rho^j)$ do not lie in the same space
\item<+-> We must interpolate them on a single space $\Omega_{fluid}^0$
\end{itemize}
%so that we have a family of snapshots belonging to the \emph{same} functional space $H:\mathcal{L}^2(\Omega_{fluid}^0)$.
\only<1>{%
\begin{center}
\begin{tabular}{|c|c|c|c|}
\hline
\includegraphics[width=0.2\linewidth, height=1.6cm]{../Figures2D/sol_1_sur8cer_un_ray.png}%
&%
\includegraphics[width=0.2\linewidth, height=1.6cm]{../Figures2D/sol_2_sur8cer_un_ray.png}%
&%
\includegraphics[width=0.2\linewidth, height=1.6cm]{../Figures2D/sol_3_sur8cer_un_ray.png}%
&%
\includegraphics[width=0.2\linewidth, height=1.6cm]{../Figures2D/sol_4_sur8cer_un_ray.png}%
\\
\hline
\includegraphics[width=0.2\linewidth, height=1.6cm]{../Figures2D/sol_5_sur8cer_un_ray.png}%
&%
\includegraphics[width=0.2\linewidth, height=1.6cm]{../Figures2D/sol_6_sur8cer_un_ray.png}%
&%
\includegraphics[width=0.2\linewidth, height=1.6cm]{../Figures2D/sol_7_sur8cer_un_ray.png}%
&%
\includegraphics[width=0.2\linewidth, height=1.6cm]{../Figures2D/sol_8_sur8cer_un_ray.png}%
\\
\hline
\end{tabular}
\end{center}
}
%
\only<2->{%
\begin{center}
\begin{tabular}{|c|c|c|c|}
\hline
\includegraphics[width=0.2\linewidth, height=1.6cm]{../Figures2D/snap_1_sur8cer_un_ray.png}%
&%
\includegraphics[width=0.2\linewidth, height=1.6cm]{../Figures2D/snap_2_sur8cer_un_ray.png}%
&%
\includegraphics[width=0.2\linewidth, height=1.6cm]{../Figures2D/snap_3_sur8cer_un_ray.png}%
&%
\includegraphics[width=0.2\linewidth, height=1.6cm]{../Figures2D/snap_4_sur8cer_un_ray.png}%
\\
\hline
\includegraphics[width=0.2\linewidth, height=1.6cm]{../Figures2D/snap_5_sur8cer_un_ray.png}%
&%
\includegraphics[width=0.2\linewidth, height=1.6cm]{../Figures2D/snap_6_sur8cer_un_ray.png}%
&%
\includegraphics[width=0.2\linewidth, height=1.6cm]{../Figures2D/snap_7_sur8cer_un_ray.png}%
&%
\includegraphics[width=0.2\linewidth, height=1.6cm]{../Figures2D/snap_8_sur8cer_un_ray.png}%
\\
\hline
\end{tabular}
\end{center}
}
%
\footnotesize{Here $\rho^j$ is a single real parameter : the radius of a disc, $j=1 \dots 8$.}
\end{block}
%
\begin{block}{Step 3 : POD on $\mathcal{L}^2(\Omega_{fluid}^0)$}<3->
\begin{itemize}
\item<+-> Compute the $(\lambda_i,\phi_i)_{i=1}^{N_{snap}}$ using matrix $\mathcal{C}$ ;%and linear algebra ;
%
%\begin{center}
%\begin{tabular}{|c|c|c|c|}
%\hline
%\includegraphics[width=0.2\linewidth, height=1.4cm]{../Figures2D/phi_1_cer_un_ray.png}%
%&%
%\includegraphics[width=0.2\linewidth, height=1.4cm]{../Figures2D/phi_2_cer_un_ray.png}%
%&%
%\includegraphics[width=0.2\linewidth, height=1.4cm]{../Figures2D/phi_3_cer_un_ray.png}%
%&%
%\includegraphics[width=0.2\linewidth, height=1.4cm]{../Figures3D/phi_4_cer_un_ray.png}%
%\\
%\hline
%\includegraphics[width=0.2\linewidth, height=1.4cm]{../Figures2D/phi_5_cer_un_ray.png}%
%&%
%\includegraphics[width=0.2\linewidth, height=1.4cm]{../Figures2D/phi_6_cer_un_ray.png}%
%&%
%\includegraphics[width=0.2\linewidth, height=1.4cm]{../Figures2D/phi_7_cer_un_ray.png}%
%&%
%\includegraphics[width=0.2\linewidth, height=1.4cm]{../Figures2D/phi_8_cer_un_ray.png}%
%\\
%\hline
%\end{tabular}
%\end{center}
%
%\item<+-> Compute $(\phi_i)_{i=1}^{N_{snap}}$ using the $a_i$ ;
\item<+-> Choose the number of modes $N_{rom}$ to be kept.%to keep for a reduced order model.
\end{itemize}
\end{block}
%
\end{frame}

\begin{frame}%{The algorithm we use : inline}
%
\begin{block}{Step 4 : set a parameter $\rho^{new}$}<+->
\begin{itemize}
\item<+-> Project the $\phi_i$ POD vectors on the new fluid domain $\Omega_f$
\item<+-> We obtain $N_{rom}$ spatial modes $\phi_i^{new}$.
\item<+-> Resolve the $N_{rom}\times N_{rom}-$linear problem to compute $a_k(\rho^{new})$ :
\end{itemize}
\visible<4->{
%\[%
%\begin{split}
%\sum\limits_{k=1}^{N_{rom}} a_k(\rho^{new})&\int\limits_{\Omega}Tr\left(\langle grad_y\phi_k^{new} \left(\mathbf{y},\rho\right)|grad_y\left(\phi_i^{new}(\mathbf{y}\right)\rangle\right)\text{d}\Omega\\%
%=-&\int\limits_{\Gamma_{sf}}\langle n_{sf}|\phi_i^{new}(\mathbf{y})\rangle \text{d}s
%\end{split}
%\]
\(
\sum\limits_{k=1}^{N_{rom}} a_k(\rho^{new})%
\int\limits_{\Omega}%
Tr\left(\langle grad_{\mathbf{y}}\phi_k^{new}|grad_{\mathbf{y}}\phi_i^{new}\rangle\right)%
\text{d}\Omega~=%
-\int\limits_{\Gamma_{sf}}\langle n_{sf}|\phi_i^{new}(\mathbf{y})\rangle \text{d}s
\)
}
\begin{itemize}
\item<+-> Deduce $\chi(\mathbf{y},\rho^{new})$ : $\chi\left(\mathbf{y},\rho^{new}\right)\simeq \sum\limits_{i=1}^{N_{rom}} a_i \left(\rho^{new}\right)\phi_i^{new}(\mathbf{y})$
\end{itemize}
\end{block}
%
\end{frame}
