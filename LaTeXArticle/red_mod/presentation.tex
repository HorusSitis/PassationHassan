%\section{Reduced Order Models}

\begin{frame}{Proper Orthogonal Decomposition}
%
\begin{block}{Motivation}<+->
%
\begin{itemize}
\item<+-> We want to compute a function of $\mathbf{x}$ : $w(\mathbf{x},\tau)$, which lie in an Hilbert space $H$ \dots
\item<+-> \dots depending on a real parameter $\tau$ : time, porosity etc.
%\item<+-> $w(\cdot ,\tau)$ lies in an Hilbert space $H$ for each $\tau$
\end{itemize}
%
\end{block}
%
\begin{block}{POD : definition}<+->
\begin{itemize}
\item %<+-> 
$w(\mathbf{x},\tau)=\sum\limits_{i=1}^{N_{POD}} a_i(\tau)\phi_i(\mathbf{x})$ for every $\tau$ ;
\item %<+-> 
$(\phi_i)_{i=1}^{N_{POD}}$ \emph{alias} $\Phi$ is an orthogonal family in $H$.
%\item<+-> The functions $a_i$ need to be computated.
\end{itemize}
\end{block}
%
\begin{block}{Computation of $\Phi$}<+->
\begin{description}
\item %<+-> 
[Data :] We know the $w(\mathbf{x},\tau_j)$ for $\tau$ in a finite set $\{\tau_j\}$ ;
\item %<+-> 
[Optimality of $\Phi$] $\phi_l= \arg\max\limits_{\phi_j\in H}\left\langle\left(\phi_j\left|\sum\limits_{k=1}^{l-1}\left(w|\phi_k\right)_H\phi_k\right.\right)_H\right\rangle$% ;
\item %<+-> 
[Fredholm equation] $\int\limits_{\Omega}R(\mathbf{x},\mathbf{x'})\phi_j (\mathbf{x})\text{d}\mathbf{x'}=\lambda_j \phi_j (x)$.
\end{description}
\end{block}
%
\end{frame}

\begin{frame}{Method of snapshots}
%
\begin{block}{}<+->
\begin{itemize}
\item<+-> We have a sample $w(\mathbf{x},\tau_j)_{j=1}^{N_{snap}}$ of spatial functions, called \emph{snapshots} \dots
\item<+-> \dots coming from experiments or numerical simulations.
\item<+-> We write : %$\phi_i(\mathbf{x})=\sum\limits_{j=1}^{N_{\tau}}\alpha_i(\tau_j) w(\mathbf{x},\tau_j)$.
$\phi_i(\mathbf{x})=\sum\limits_{j=1}^{N_{snap}}a_i(\tau_j) w(\mathbf{x},\tau_j)$
\end{itemize}
\end{block}
%
\begin{block}{}<+->%The new POD problem}<+->%Fredholm equation}<+->
\begin{description}
\item %<+-> 
[Fredholm equation]
\[%
%\begin{split}&
\iint\alpha_i(\tau)C(\tau,\tau')w(\mathbf{x},\tau ')\text{d}\tau'\text{d}\tau%\\%
%=&
=\lambda_i\int\alpha_i(\tau')w(\mathbf{x},\tau')\text{d}\tau'%
%\end{split}%
\]
\item %<+-> 
[Time-correlation tensor] $C(\tau,\tau ')=\dfrac{1}{N_{\tau}}\int\limits_{\Omega} w(\mathbf{x},\tau)w(\mathbf{x},\tau')\text{d}\mathbf{x}$
\item %<+-> 
[Eigenvalue problem] %$\mathcal{C}_{\tau,\tau'}\alpha_i(\tau)=\lambda_i\alpha_i(\tau)$
$\mathcal{C}_{\tau,\tau'}a_i(\tau)=\lambda_i a_i(\tau)$
%\item<+-> [POD] $\alpha_i$ are the $a_i$ in POD.
\end{description}
\end{block}
%
\end{frame}

\begin{frame}{Reduced Order Model : general case}
%
\begin{block}{Reduced basis}<+->
\begin{itemize}
\item<+-> We are given a POD basis $(\phi_i)_{i=1}^{N_{snap}}$ with a set $\lambda_i$ of associated proper-values ;
%\item<+-> $N_{POD}$ is the number of snapshots when we use the so-called mehtod ;
\item<+-> We choose the first $N_{rom}$ vectors of $\Phi$ so that : %
$\dfrac{Ener_{N_{rom}}}{Ener_{N_{snap}}}\text{ \textit{id.est} }\dfrac{\sum\limits_{k=1}^{N_{rom}}\lambda_k}{\sum\limits_{k=1}^{N_{snap}}\lambda_k} >99,9\dots \%$ ;
\item<+-> $N_{rom}$ must be small.
\end{itemize}
\end{block}
%
\end{frame}

\begin{frame}{Reduced Order Model : problem solving}
\begin{block}{Galerkin projection}<+->
%
\begin{description}
\item<+-> [For a given problem] $\mathcal{A}^{\mathbf{x},\tau}\left(w(\mathbf{x},\tau)\right)=f(\mathbf{x},\tau)$
\item<+-> [Decomposition] $w(\mathbf{x},\tau)=\sum\limits_{i=1}^N a_i(\tau)\phi_i(\mathbf{x})$ ;% +\mathcal{R}_{N_{rom}}(\mathbf{x},\tau)$ , with $\mathcal{R}_{N_{rom}}$ orthogonal to $(\phi_i)_{i=1}^{N_{rom}}$ ;
\item<+-> [The problem in $a_k$] %
\[%
\begin{split}
\left(\left.\mathcal{A}\left(\sum\limits_{k=1}^{N_{rom}} a_k(\tau)\phi_k(\mathbf{x})\right) \right|\phi_i\left(\mathbf{x}\right)\right)_H %
&= \left(f\left(\mathbf{x},\tau\right)|\phi_i(\mathbf{x})\right)_H\\%
&+\mathcal{R}_{N_{rom}}(\mathbf{x},\tau)
\end{split}
\]
\item<+-> [\dots] which is easier to resolve than the original problem.
\end{description}
%
\end{block}
%
\end{frame}

\begin{frame}{The algorithm we use : offline}
%
\begin{block}{Step 1 : generation of snapshots}<+->
\begin{itemize}
\item<+-> Using Finite Element Method, compute $\chi(\mathbf{y},\rho_j)$ for a finite number of $\rho_j$ ;
\item<+-> Interpolate the solutions onto a single space $\Omega_{fluid}^0$, %
so that we have a family of snapshots belonging to the \emph{same} functional space $H$.
\end{itemize}
\end{block}
%
\begin{block}{Step 2 : POD}<+->
\begin{itemize}
\item<+-> Compute $\mathcal{C}$ and find the $a_i(\rho_j)$ using linear algebra ;
\item<+-> Compute $(\phi_i)_{i=1}^{N_{snap}}$ using the $a_i$ ;
\item<+-> Choose the number of modes $N_{rom}$ we keep for a reduced order model.
\end{itemize}
\end{block}
%
\end{frame}

\begin{frame}{The algorithm we use : inline}
%
\begin{block}{Step 3 : ROM with given parameter $\rho^{new}$}<+->
\begin{itemize}
\item<+-> Extrapolate the $\phi_i$ POD vectors to a family $(\phi_i^{new})$ of functions defined for $\mathbf{y}$ in $\Omega_{fluid}^{new}$ ;
\item<+-> Resolve the $N_{rom}\times N_{rom}-$linear problem in $a_k(\rho^{new})$ :
\end{itemize}
\visible<3->{
\[%
\begin{split}
\sum\limits_{k=1}^{N_{rom}} a_k(\rho^{new})&\int\limits_{\Omega}Tr\left(\langle grad_y\phi_k^{new} \left(\mathbf{y},\rho\right)|grad_y\left(\phi_i^{new}(\mathbf{y}\right)\rangle\right)\text{d}\Omega\\%
=-&\int\limits_{\Gamma_{sf}}\langle n_{sf}|\phi_i^{new}(\mathbf{y})\rangle \text{d}s
\end{split}
\]
}
\begin{itemize}
\item<+-> Compute $\chi(\mathbf{y},\rho^{new})$ : $\chi\left(\mathbf{y},\rho^{new}\right)\simeq \sum\limits_{i=1}^{N_{rom}} a_i \left(\rho^{new}\right)\phi_i^{new}(\mathbf{y})$
\end{itemize}
\end{block}
%
\end{frame}
