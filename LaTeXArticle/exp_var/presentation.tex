%\section{Multiscale structure}

\begin{frame}{Miscellaneous images}%{Observing cementarious material}
%
%\begin{block}{}<+->%{Microtomography, ESEM}<+->
\begin{multicols}{2}
Cementarious media seen with microtomography
%
\includegraphics[width=\linewidth,height=5cm]{../Figures2D/scales_obs.png}

\par
\footnotesize{Chalen\c con et al., 2009}

\columnbreak
Multiple scale for heterogeneities : 

\includegraphics[width=\linewidth, height=6cm]{../Figures2D/mulscBoltzMZhang.png}

\par
\footnotesize{Zhang, M. 2013}
\end{multicols}
%\end{block}
%
%\begin{block}{Scanning electron migroscope}<+->%{Scanning Electron Microscope}<+->
%\includegraphics[width=\linewidth,height=2cm]{../Figures2D/exp_var/meb_cem.png}
%Samples of cementarious medium.
%\end{block}
%
\end{frame}

\begin{frame}{Consequences for practical purpose}
%
\begin{block}{Experiment}<+->
\begin{itemize}
\item<+-> The material's microstructure is variable
\item<+-> We would need a huge number of observations
\item<+-> Tomography, FIB, MEB measures are expensive
\end{itemize}
\end{block}
%
\begin{block}{Computation}<+->
\begin{description}
\item<+-> [Objectives] Resolve Stokes or Nernst-Planck equations
\item<+-> [High spatial resolution] 1000$\times$1000$\times$1000 voxels
\item<+-> [Small sample] A solution can't be extrapolated to the whole structure
\end{description}
\end{block}
%
\end{frame}