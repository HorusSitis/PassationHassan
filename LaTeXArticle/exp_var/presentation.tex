%\section{Multiscale structure}

\begin{frame}{Observing cementarious material}
%
\begin{block}{}<+->%{Microtomography, ESEM}<+->
\begin{multicols}{2}
\includegraphics[width=\linewidth,height=3.8cm]{exp_var/tom_esem.png}

\columnbreak
Cementarious media seen with microtomography

\par
We can see a heterogeneous structure
\end{multicols}
\end{block}
%
\begin{block}{}<+->%{Scanning Electron Microscope}<+->
\begin{multicols}{2}
\includegraphics[width=\linewidth,height=2cm]{exp_var/meb_cem.png}

\columnbreak
Samples of cementarious medium, seen with Scanning Electron Microscope

\par
We still see heterogeneities
\end{multicols}
\end{block}
%
\begin{block}{Observations}<+->
%\begin{itemize}
%\item<+-> 
Heterogeneities appear at diffrerent scales : mm to nm ;
%\item<+-> 
%\end{itemize}
\end{block}
%
\end{frame}

\begin{frame}{Issues for practical purpose}
%
\begin{block}{Experiment}<+->
\begin{itemize}
\item<+-> Measurement : tomography, BEM \dots are expensive ;
\item<+-> Small samples of matter ;
\item<+-> We would need a huge number of observations.
\end{itemize}
\end{block}
%
\begin{block}{Computation}<+->
\begin{itemize}
\item<+-> Solving PDE with high spatial resolution ;
\item<+-> 1000$\times$1000$\times$1000 voxels ;
\item<+-> A particular numerical solution cannot be used for a given sample of concrete.
\end{itemize}
\end{block}
%
\begin{block}{Objectives}<+->
\begin{itemize}
\item<+-> Find representative samples of concrete structure, and measure some physical parameters ;
\item<+-> Resolve numerically a PDE on it ;
\item<+-> Varying the parameter, solve quickly the PDE on another sample using a Reduced Order Model.
\end{itemize}
\end{block}
%
\end{frame}