%\section{Multiscale structure}

\begin{frame}{Observing cementarious material}
%
\begin{block}{}<+->%{Microtomography, ESEM}<+->
\begin{multicols}{2}
\includegraphics[width=\linewidth,height=3.8cm]{exp_var/tom_esem.png}

\columnbreak
Cementarious media seen with microtomography

\par
We can see a heterogeneous structure
\end{multicols}
\end{block}
%
\begin{block}{}<+->%{Scanning Electron Microscope}<+->
\begin{multicols}{2}
\includegraphics[width=\linewidth,height=2cm]{exp_var/meb_cem.png}

\columnbreak
Samples of cementarious medium, seen with Scanning Electron Microscope

\par
We still see heterogeneities
\end{multicols}
\end{block}
%
\begin{block}{Observations}<+->
%\begin{itemize}
%\item<+-> 
Heterogeneities appear at diffrerent scales : mm to nm ;
%\item<+-> 
%\end{itemize}
\end{block}
%
\end{frame}

\begin{frame}{Consequences for practical purpose}
%
\begin{block}{Experiment}<+->
\begin{itemize}
\item<+-> The material's microstructure is variable ;
\item<+-> We would need a huge number of observations ;
\item<+-> Tomography, FIB, MEB measures are expensive.
\end{itemize}
\end{block}
%
\begin{block}{Computation}<+->
\begin{description}
\item<+-> [Objectives] Resolve Navier-Stokes, diffusion equations.
\item<+-> [High spatial resolution] 1000$\times$1000$\times$1000 voxels ;
\item<+-> [Small sample] A solution can't be extrapolated to the whole structure.
\end{description}
\end{block}
%
\begin{block}{Method for problem solving}<+->
\begin{itemize}
%\item<+-> Find representative samples of concrete structure, and measure some physical parameters ;
%\item<+-> Resolve numerically the problem on it ;
%\item<+-> Varying the parameter, solve quickly the PDE on another sample using a Reduced Order Model.
\item<+-> Characterize the microstructure with some parameters : porosity etc ;
\item<+-> Solve the problem for a few parameter values ;
\item<+-> Interpolate the solutions, varying the parameters.
\end{itemize}
\end{block}
%
\end{frame}