\section{Results and discussion}\label{res}

\subsection{Material}

\begin{description}
\item [Machine] 1$\times$CPU 2.90GHz
\item [FEM] Resolution of the weak problem in $\chi$ was done with FEniCS, %
an open-source platform programmed in Python designed for Finite Element Method.
\item [ROM] The numpy package of python 3 was used for POD.
\end{description}

\subsection{Two-dimensional case : varying the radius of a single circular inclusion}
%\subsection{Two-dimensional case : varying the radius of a single circular inclusion}

For all the two-dimensional cases, resolution is set to $100$ : %
100 nodes lie on each side of the unit square.

\par
For Finite Element Method, in the training step and for validation of the Reduced Order Model, we chose polynomial elements of degree $2$.

\subsubsection{Centered inclusion}

See section \ \ref{rom} for figures and choice of $N_{rom}$.
%\begin{comment}
\begin{figure}[H]%{Results}
%
\begin{center}
\begin{tabular}{|c|c||c|c||c|c||c||c|c||c||c|}
\hline
\rowcolor{lightgray} $\rho^{new}$&Porosity&${D_k^{hom,ROM}}_{11}$&${D_k^{hom,FEM}}_{11}$&Meshing&$Err$&$\phi_i^{new}$&ROM&FEM&Nodes\\
\hline
0,11&0,9620&0,9284&0,9268&1.84s&0,1753\%&2.25s&0.34s&10.64s&103090\\
\hline
0,22&0,8479&0,7362&0,7360&1.23s&0,0232\%&1.83s&0.45s&7.26s&90062\\
\hline
0,33&0,6579&0,488596&0,4488589&1.00s&0,0014\%&1.48s&0.31s&4.14s&70118\\
\hline
0,44&0,3918&0,2228&0,2227&0.85s&0,0702\%&1.00s&0.22s&1.48s&41458\\
\hline
\end{tabular}
\end{center}
\caption{$N_{rom}=2$}
%
\end{figure}

%$99,9 \% : N_{rom}=4$.
%\end{comment}
\subsubsection{Disc at vertices}

\begin{figure}[H]
\begin{center}
\begin{tabular}{|c|c|c|c|}
\hline
\subfloat[$D_k^{hom}=0.9845$]{\includegraphics[width=0.2\linewidth, height=2.2cm]{../Figures2D/sol_1_sur8cer_un_som_ray.png}}%
&%
\subfloat[$D_k^{hom}=0.9391$]{\includegraphics[width=0.2\linewidth, height=2.2cm]{../Figures2D/sol_2_sur8cer_un_som_ray.png}}%
&%
\subfloat[$D_k^{hom}=0.8680$]{\includegraphics[width=0.2\linewidth, height=2.2cm]{../Figures2D/sol_3_sur8cer_un_som_ray.png}}%
&%
\subfloat[$D_k^{hom}=0.7767$]{\includegraphics[width=0.2\linewidth, height=2.2cm]{../Figures2D/sol_4_sur8cer_un_som_ray.png}}%
\\
\hline
\subfloat[$D_k^{hom}=0.6716$]{\includegraphics[width=0.2\linewidth, height=2.2cm]{../Figures2D/sol_5_sur8cer_un_som_ray.png}}%
&%
\subfloat[$D_k^{hom}=0.5585$]{\includegraphics[width=0.2\linewidth, height=2.2cm]{../Figures2D/sol_6_sur8cer_un_som_ray.png}}%
&%
\subfloat[$D_k^{hom}=0.4415$]{\includegraphics[width=0.2\linewidth, height=2.2cm]{../Figures2D/sol_7_sur8cer_un_som_ray.png}}%
&%
\subfloat[$D_k^{hom}=0.3221$]{\includegraphics[width=0.2\linewidth, height=2.2cm]{../Figures2D/sol_8_sur8cer_un_som_ray.png}}%
\\
\hline
\end{tabular}
\end{center}
\caption{The vector fields $\chi(\mathbf{x},\rho_j )$}
\end{figure}

\begin{figure}[H]
\begin{center}
\begin{tabular}{|c|c|}
\hline
\subfloat[$\phi_1$]{\includegraphics[width=0.2\linewidth, height=2.2cm]{../Figures2D/phi_1_compl_hor.png}}%
&%
\subfloat[$\phi_2$]{\includegraphics[width=0.2\linewidth, height=2.2cm]{../Figures2D/phi_2_compl_hor.png}}%
\\
\hline
\end{tabular}
\end{center}
\caption{The truncated POD basis}
\end{figure}

\begin{figure}[H]
\begin{center}
\begin{tabular}{|c|c|}
\hline
\includegraphics[width=0.4\linewidth, height=3cm]{../Figures2D/ener_vp_cer_un_som_ray.png}%_res100
&%
\includegraphics[width=0.4\linewidth, height=3cm]{../Figures2D/ener_cumul_vp_cer_un_som_ray.png}%_res10
\\ \hline
\end{tabular}
\end{center}
\caption{Energy of POD modes}
\end{figure}

\begin{multicols}{2}
Cumulated energy :

\columnbreak
\begin{itemize}
\item $99,9 \% : N_{rom}=1$ ;
\item $99,99 \% : N_{rom}=2$.
\end{itemize}
%
\end{multicols}

\begin{figure}[H]%{Results}
%
\begin{center}
\begin{tabular}{|c|c||c|c||c||c|c||c|c||c||c|}
\hline
\rowcolor{lightgray} $\rho^{new}$&Porosity&${D_k^{hom,ROM}}_{11}$&${D_k^{hom,FEM}}_{11}$&Meshing&$Err$&$\phi_i^{new}$&ROM&FEM&Nodes\\
\hline
0,11&0,9620&0,9285&0,9268&1.84s&0,03670.1852\%&2.13s&0.33s&11.71s&101902\\
\hline
0,22&0,8479&0,7362&0,7360&1.23s&0,0242\%&2.07s&0.35s&7.75s&91606\\
\hline
0,33&0,6579&0,488596&0,488589&1.02s&0,00143\%&1.49s&0.53s&3.99s&70518\\
\hline
0,44&0,3918&0,2228&0,2227&0.77s&0,0702\%&1.02s&0.21s&1.69s&43118\\
\hline
\end{tabular}
\end{center}
\caption{$N_{rom}=2$}
%
\end{figure}

Here again, time difference between using Reduced Order Model instead of Finite Element Method is sensitive, %
for small values of $\rho$ which is a decreasinf function of porosity.

\par
The difference of time ratio FEM/ROM will depend from the number of nodes, as we can see in following section about three-dimensional geometries. %
 
\subsubsection{Diagonal alignement}

\begin{figure}[H]
\begin{center}
\begin{tabular}{|c|c|c|c|}
\hline
\subfloat[$D_k^{hom}=0.8595$]{\includegraphics[width=0.2\linewidth, height=2.2cm]{../Figures2D/sol_1_sur8compl_diag.png}}%
&%
\subfloat[$D_k^{hom}=0.8196$]{\includegraphics[width=0.2\linewidth, height=2.2cm]{../Figures2D/sol_2_sur8compl_diag.png}}%
&%
\subfloat[$D_k^{hom}=0.7658$]{\includegraphics[width=0.2\linewidth, height=2.2cm]{../Figures2D/sol_3_sur8compl_diag.png}}%
&%
\subfloat[$D_k^{hom}=0.6758$]{\includegraphics[width=0.2\linewidth, height=2.2cm]{../Figures2D/sol_4_sur8compl_diag.png}}%
\\
\hline
\subfloat[$D_k^{hom}=0.5819$]{\includegraphics[width=0.2\linewidth, height=2.2cm]{../Figures2D/sol_5_sur8compl_diag.png}}%
&%
\subfloat[$D_k^{hom}=0.4806$]{\includegraphics[width=0.2\linewidth, height=2.2cm]{../Figures2D/sol_6_sur8compl_diag.png}}%
&%
\subfloat[$D_k^{hom}=0.3762$]{\includegraphics[width=0.2\linewidth, height=2.2cm]{../Figures2D/sol_7_sur8compl_diag.png}}%
&%
\subfloat[$D_k^{hom}=0.2714$]{\includegraphics[width=0.2\linewidth, height=2.2cm]{../Figures2D/sol_8_sur8compl_diag.png}}%
\\
\hline
\end{tabular}
\end{center}
\caption{The vector fields $\chi(\mathbf{x},\rho_j )$}
\end{figure}

\begin{figure}[H]
\begin{center}
\begin{tabular}{|c|c|c|}
\hline
\subfloat[$\phi_1$]{\includegraphics[width=0.2\linewidth, height=2.2cm]{../Figures2D/phi_1_compl_diag.png}}%
&%
\subfloat[$\phi_2$]{\includegraphics[width=0.2\linewidth, height=2.2cm]{../Figures2D/phi_2_compl_diag.png}}%
&%
\subfloat[$\phi_3$]{\includegraphics[width=0.2\linewidth, height=2.2cm]{../Figures2D/phi_3_compl_diag.png}}%
\\
\hline
\end{tabular}
\end{center}
\caption{The truncated POD basis}
\end{figure}

\begin{figure}[H]
\begin{center}
\begin{tabular}{|c|c|}
\hline
\includegraphics[width=0.4\linewidth, height=3cm]{../Figures2D/ener_vp_compl_diag.png}%_res100
&%
\includegraphics[width=0.4\linewidth, height=3cm]{../Figures2D/ener_cumul_vp_compl_diag.png}%_res10
\\ \hline
\end{tabular}
\end{center}
\caption{Energy of POD modes}
\end{figure}

\begin{figure}[H]%{Results}
%
\begin{center}
\begin{tabular}{|c|c||c|c||c||c|c||c|c||c||c|}
\hline
\rowcolor{lightgray} $\rho^{new}$&Porosity&${D_k^{hom,ROM}}_{11}$&${D_k^{hom,FEM}}_{11}$&Meshing&$Err$&$\phi_i^{new}$&ROM&FEM&Nodes\\
\hline
0,11&0,8913&0,8090&0,8087&1.80s&0,0367\%&3.71s&1.09s&7.86s&96184\\
\hline
0,22&0,7773&0,6396&0,6395&1.24s&0,0195\%&2.95s&0.33s&5.45s&83020\\
\hline
0,33&0,5872&0,41807206&0,41807209&0.94s&6.44$\times 10^{-6}$\%&2.23s&0.30s&2.95s&23764\\
\hline
0,44&0,3211&0,1658769&0,1658763&0.77s&0,00033\%&1.58s&0.42s&1.77s&17054\\
\hline
\end{tabular}
\end{center}
\caption{$N_{rom}=3$}
%
\end{figure}

\subsubsection{Horizontal alignement}

\begin{figure}[H]
\begin{center}
\begin{tabular}{|c|c|c|c|}
\hline
\subfloat[$D_k^{hom}=0.8582$]{\includegraphics[width=0.2\linewidth, height=2.2cm]{../Figures2D/sol_1_sur8compl_hor.png}}%
&%
\subfloat[$D_k^{hom}=0.8365$]{\includegraphics[width=0.2\linewidth, height=2.2cm]{../Figures2D/sol_2_sur8compl_hor.png}}%
&%
\subfloat[$D_k^{hom}=0.8028$]{\includegraphics[width=0.2\linewidth, height=2.2cm]{../Figures2D/sol_3_sur8compl_hor.png}}%
&%
\subfloat[$D_k^{hom}=0.7578$]{\includegraphics[width=0.2\linewidth, height=2.2cm]{../Figures2D/sol_4_sur8compl_hor.png}}%
\\
\hline
\subfloat[$D_k^{hom}=0.7020$]{\includegraphics[width=0.2\linewidth, height=2.2cm]{../Figures2D/sol_5_sur8compl_hor.png}}%
&%
\subfloat[$D_k^{hom}=0.6364$]{\includegraphics[width=0.2\linewidth, height=2.2cm]{../Figures2D/sol_6_sur8compl_hor.png}}%
&%
\subfloat[$D_k^{hom}=0.5619$]{\includegraphics[width=0.2\linewidth, height=2.2cm]{../Figures2D/sol_7_sur8compl_hor.png}}%
&%
\subfloat[$D_k^{hom}=0.4796$]{\includegraphics[width=0.2\linewidth, height=2.2cm]{../Figures2D/sol_8_sur8compl_hor.png}}%
\\
\hline
\end{tabular}
\end{center}
\caption{The vector fields $\chi(\mathbf{x},\rho_j )$}
\end{figure}


\begin{figure}[H]
\begin{center}
\begin{tabular}{|c|c|c|c|}
\hline
\subfloat[$\phi_1$]{\includegraphics[width=0.2\linewidth, height=2.2cm]{../Figures2D/phi_1_compl_hor.png}}%
&%
\subfloat[$\phi_2$]{\includegraphics[width=0.2\linewidth, height=2.2cm]{../Figures2D/phi_2_compl_hor.png}}%
&%
\subfloat[$\phi_3$]{\includegraphics[width=0.2\linewidth, height=2.2cm]{../Figures2D/phi_3_compl_hor.png}}%
&%
\subfloat[$\phi_4$]{\includegraphics[width=0.2\linewidth, height=2.2cm]{../Figures2D/phi_4_compl_hor.png}}%
\\
\hline
\end{tabular}
\end{center}
\caption{The truncated POD basis}
\end{figure}

\begin{figure}[H]
\begin{center}
\begin{tabular}{|c|c|}
\hline
\includegraphics[width=0.4\linewidth, height=3cm]{../Figures2D/ener_vp_compl_hor.png}%_res100
&%
\includegraphics[width=0.4\linewidth, height=3cm]{../Figures2D/ener_cumul_vp_compl_hor.png}%_res10
\\ \hline
\end{tabular}
\end{center}
\caption{Energy of POD modes}
\end{figure}


\begin{figure}[H]%{Results}
%
\begin{center}
\begin{tabular}{|c|c||c|c||c||c|c||c|c||c||c|}
\hline
\rowcolor{lightgray} $\rho^{new}$&Porosity&${D_k^{hom,ROM}}_{11}$&${D_k^{hom,FEM}}_{11}$&Meshing&$Err$&$\phi_i^{new}$&ROM&FEM&Nodes\\
\hline
0,04&0,9243&0,8683&0,8618&1.98s&0,7592\%&4.03s&0.47s&9.52s&99\ 784\\
\hline
0,10&0,8979&0,82924&0,82921&1.36s&0,00356\%&3.85s&0.46s&7.97s&96\ 844\\
\hline
0,20&0,8037&0,71693&0,71694&1.21s&0,00079\%&3.92s&0.41s&5.66s&86\ 184\\
\hline
0,30&0,6466&0,5414&0,5420&1.14s&0,1021\%&3.15s&0.38s&3.76s&68\ 840\\
\hline
\end{tabular}
\end{center}
\caption{$N_{rom}=4$}
%
\end{figure}


%\Floatbarrier
\subsection{Three-dimensional case}

Following geometries, especially when the number of nodes in the mesh where Finite Element vector space is defined, fully justify the use of a Reduced Order Model, %
sice time of execution of the Finite Element Method becomes significant.

\par
Like in section \ref{hom_d}, we have chosen three different resolutions of the meshes used for each problem to solve with a ROM. %
These are defined by the number of nodes : $10$, $20$ or $50$, we find on a side of the unit cube.

\par
The following table gives the values of $N_{rom}$ with a threshold of $99,99\%$. We can see that, for a same problem in $\chi (\rho ,\mathbf{y})$, %
$N_{rom}$ depends on the size of the mesh : %
this can be understood mathematically since it has been defined, and calculated, from a Finite Element space. %
The latter is nothing more than an approximation of $L^2$ or $H^1$ we use for our physical problem.

\par
Therefore the physically relevant $N_{rom}$, which might be equal to the mathematical $N_{rom}$ for $L^2$ or $H^1$ functional space, %
is likely to be different of the values we find in the following table, %
and we can reasonably thik it will be lesser than its value for the biggest mesh.

\par
For example, $N_{rom}$ for a physical, spherical inclusion will be lesser than $3$, and lesser than $2$ for two spheres per unit cell.

\par
We should consider this issue since $N_{rom}$ may not be too small, if we want ROM to be relevant. %
Indeed, the trivial case when $N_{rom}$ equals $1$ leads us to choose, according to POD's definition, %
the mean value the interpolated $\chi '(,\rho_j)$ for the single POD vector.

\par
Hence resolution of the optimisation problem, directly or with method of snapshots, would be no longer useful.

{\renewcommand{\tabularxcolumn}[1]{%
>{\centering\arraybackslash}m{#1}}

\begin{figure}[H]
%
\begin{center}
%\begin{tabular}{|l||*{5}{c|}}
\begin{tabularx}{0.9\linewidth}%
{|>{\bfseries}c||%
*{3}{c|}%
*{2}{>{\centering \arraybackslash}X|}}
%
\hline
\rowcolor{lightgray} %
Res.%\backslashbox{Res.}{Config.}%
&Single sphere&Single cylinder&Two spheres&%
One sphere, one cylinder, $\rho=$ sphere radius&One sphere, one cylinder, $\rho=$ cylinder radius\\
%
\hline
\hline
10&5&5&4&4&5\\ \hline
20&4&5&3&3&4\\ \hline
50&3&5&2&2&4\\ \hline
\end{tabularx}
\end{center}
%
\caption{Values of $N_{rom}^{99,99\%}$, against the numer of nodes per unit segment}
\end{figure}

%%%%%%%%%%%%%%%%%%%%%%%%%%%%%%%%%%%%%%%%%%%%%%%%%%%%%%%%%%%%%%%%%%%%%%%%%%%%%%%%%%%%%%%%%%%%%%%%%%%%%%%%%%%%%%%%%%%%%%%%%%%%%%%%%%%%%%%%%%%%%%%%%%%%%%
%%%%%%%%%%%%%%%%%%%%%%%%%%%%%%%%%%%%%%%%%%%%%%%%%%%%%%% Résultats et figures de la dimension 3 %%%%%%%%%%%%%%%%%%%%%%%%%%%%%%%%%%%%%%%%%%%%%%%%%%%%%%%
%%%%%%%%%%%%%%%%%%%%%%%%%%%%%%%%%%%%%%%%%%%%%%%%%%%%%%%%%%%%%%%%%%%%%%%%%%%%%%%%%%%%%%%%%%%%%%%%%%%%%%%%%%%%%%%%%%%%%%%%%%%%%%%%%%%%%%%%%%%%%%%%%%%%%%
%\subsection{Three-dimensional case}

Following geometries, especially when the number of nodes in the mesh where Finite Element vector space is defined, fully justify the use of a Reduced Order Model, %
sice time of execution of the Finite Element Method becomes significant.

\par
Like in section \ref{hom_d}, we have chosen three different resolutions of the meshes used for each problem to solve with a ROM. %
These are defined by the number of nodes : $10$, $20$ or $50$, we find on a side of the unit cube.

\par
The following table gives the values of $N_{rom}$ with a threshold of $99,99\%$. We can see that, for a same problem in $\chi (\rho ,\mathbf{y})$, %
$N_{rom}$ depends on the size of the mesh : %
this can be understood mathematically since it has been defined, and calculated, from a Finite Element space. %
The latter is nothing more than an approximation of $L^2$ or $H^1$ we use for our physical problem.

\par
Therefore the physically relevant $N_{rom}$, which might be equal to the mathematical $N_{rom}$ for $L^2$ or $H^1$ functional space, %
is likely to be different of the values we find in the following table, %
and we can reasonably thik it will be lesser than its value for the biggest mesh.

\par
For example, $N_{rom}$ for a physical, spherical inclusion will be lesser than $3$, and lesser than $2$ for two spheres per unit cell.

\par
We should consider this issue since $N_{rom}$ may not be too small, if we want ROM to be relevant. %
Indeed, the trivial case when $N_{rom}$ equals $1$ leads us to choose, according to POD's definition, %
the mean value the interpolated $\chi '(,\rho_j)$ for the single POD vector.

\par
Hence resolution of the optimisation problem, directly or with method of snapshots, would be no longer useful.

{\renewcommand{\tabularxcolumn}[1]{%
>{\centering\arraybackslash}m{#1}}

\begin{figure}[H]
%
\begin{center}
%\begin{tabular}{|l||*{5}{c|}}
\begin{tabularx}{0.9\linewidth}%
{|>{\bfseries}c||%
*{3}{c|}%
*{2}{>{\centering \arraybackslash}X|}}
%
\hline
\rowcolor{lightgray} %
Res.%\backslashbox{Res.}{Config.}%
&Single sphere&Single cylinder&Two spheres&%
One sphere, one cylinder, $\rho=$ sphere radius&One sphere, one cylinder, $\rho=$ cylinder radius\\
%
\hline
\hline
10&5&5&4&4&5\\ \hline
20&4&5&3&3&4\\ \hline
50&3&5&2&2&4\\ \hline
\end{tabularx}
\end{center}
%
\caption{Values of $N_{rom}^{99,99\%}$, against the numer of nodes per unit segment}
\end{figure}

\ligneinter
In we following, we give results and perforlance of the method for two different resolutions : %
$20$ and $50$ nodes per unit segment, %
which mean respectively steps of $0.05$ and $0.02$ for the Finite Element spaces.

\par
The case of a step of $0.1$ gives results that are close to those from meshes with step $0.05$ and is therfore irrelevant for discussion on effociency of the method.

\par
However we kept values found for $N_{rom}^{99,99\%}$ to see its evolution with respect to the number of nodes.

\subsubsection{A single spherical inclusion}

\begin{figure}[H]%[!p]
%
\begin{center}
\begin{tabular}{|c|c|c|c|}
\hline
\includegraphics[width=0.2\linewidth, height=2.2cm]{../Figures3D/sol_1_sur8sph_un_rayres20.png}%
&%
\includegraphics[width=0.2\linewidth, height=2.2cm]{../Figures3D/sol_2_sur8sph_un_rayres20.png}%
&%
\includegraphics[width=0.2\linewidth, height=2.2cm]{../Figures3D/sol_3_sur8sph_un_rayres20.png}%
&%
\includegraphics[width=0.2\linewidth, height=2.2cm]{../Figures3D/sol_4_sur8sph_un_rayres20.png}%
\\
\hline
\includegraphics[width=0.2\linewidth, height=2.2cm]{../Figures3D/sol_5_sur8sph_un_rayres20.png}%
&%
\includegraphics[width=0.2\linewidth, height=2.2cm]{../Figures3D/sol_6_sur8sph_un_rayres20.png}%
&%
\includegraphics[width=0.2\linewidth, height=2.2cm]{../Figures3D/sol_7_sur8sph_un_rayres20.png}%
&%
\includegraphics[width=0.2\linewidth, height=2.2cm]{../Figures3D/sol_8_sur8sph_un_rayres20.png}%
\\
\hline
\end{tabular}
\end{center}
\caption{Resolution : $20$\ ; $8$ training solutions}
%
\end{figure}

\begin{figure}[H]
\begin{center}
\begin{tabular}{|c|c|c|c|}
\hline
\includegraphics[width=0.2\linewidth, height=2.2cm]{../Figures3D/snap_interpam_1_sur8sph_un_rayres20.png}%snap_interpam_6_sur8sph_un_rayres20.png
&%
\includegraphics[width=0.2\linewidth, height=2.2cm]{../Figures3D/snap_interpam_2_sur8sph_un_rayres20.png}%
&%
\includegraphics[width=0.2\linewidth, height=2.2cm]{../Figures3D/snap_interpam_3_sur8sph_un_rayres20.png}%
&%
\includegraphics[width=0.2\linewidth, height=2.2cm]{../Figures3D/snap_interpam_4_sur8sph_un_rayres20.png}%
\\
\hline
\includegraphics[width=0.2\linewidth, height=2.2cm]{../Figures3D/snap_interpam_5_sur8sph_un_rayres20.png}%
&%
\includegraphics[width=0.2\linewidth, height=2.2cm]{../Figures3D/snap_interpam_6_sur8sph_un_rayres20.png}%
&%
\includegraphics[width=0.2\linewidth, height=2.2cm]{../Figures3D/snap_interpam_7_sur8sph_un_rayres20.png}%
&%
\includegraphics[width=0.2\linewidth, height=2.2cm]{../Figures3D/snap_interpam_8_sur8sph_un_rayres20.png}%
\\
\hline
\end{tabular}
\end{center}
\caption{Resolution : $20$\ ; extrapolated snapshots}
\end{figure}

\begin{figure}[H]
\begin{center}
\begin{tabular}{|c|c|}
\hline
\includegraphics[width=0.4\linewidth, height=3cm]{../Figures3D/ener_vp_sph_un_ray_res20.png}
&%
\includegraphics[width=0.4\linewidth, height=3cm]{../Figures3D/ener_cumul_vp_sph_un_ray_res20.png}
\\ \hline
\end{tabular}
\end{center}
\caption{Energy of POD modes : $N_{rom}^{99,99\%}=4$}
\end{figure}

\begin{figure}[H]%<+->
%
\begin{center}
\begin{tabular}{|c|c|c|c|}
\hline
\includegraphics[width=0.2\linewidth, height=2.2cm]{../Figures3D/phi_1_sph_un_ray_res20.png}%
&%
\includegraphics[width=0.2\linewidth, height=2.2cm]{../Figures3D/phi_2_sph_un_ray_res20.png}%
&%
\includegraphics[width=0.2\linewidth, height=2.2cm]{../Figures3D/phi_3_sph_un_ray_res20.png}%
&%
\includegraphics[width=0.2\linewidth, height=2.2cm]{../Figures3D/phi_4_sph_un_ray_res20.png}%
\\
\hline
\end{tabular}
\end{center}
%
\caption{Truncated POD basis on $\Omega_{fluid}^0$ : the unit cube}
\end{figure}
%%

\begin{figure}[H]%<+->
%
\begin{center}
\begin{tabular}{|c|c|c|c|}
\hline
\includegraphics[width=0.2\linewidth, height=2.2cm]{../Figures3D/sol_rom11_sur8sph_un_rayres20.png}%
&%
\includegraphics[width=0.2\linewidth, height=2.2cm]{../Figures3D/sol_rom22_sur8sph_un_rayres20.png}%
&%
\includegraphics[width=0.2\linewidth, height=2.2cm]{../Figures3D/sol_rom33_sur8sph_un_rayres20.png}%
&%
\includegraphics[width=0.2\linewidth, height=2.2cm]{../Figures3D/sol_rom44_sur8sph_un_rayres20.png}%
\\
\hline
\end{tabular}
\end{center}
%
\caption{Solution with reduced order model on $\Omega_{fluid}^{new}$ : $20$ nodes per unit segment}
\end{figure}

\begin{figure}[H]%{Results}
%
\begin{center}
\begin{tabular}{|c|c||c|c||c|c||c|c||c||c|}
\hline
\rowcolor{lightgray} $\rho^{new}$&Porosity&${D_k^{hom,ROM}}_{11}$&${D_k^{hom,FEM}}_{11}$&Meshing&$Err$&$\phi_i^{new}$&ROM&FEM&Nodes\\
\hline
0,11&0,9944&0,9921&0,9918&2.24s&0,024\%&53.27s&2.92s&19.96s&136\ 002\\
\hline
0,22&0,9554&0,93498&0,93495&1.77s&0,034\%&48.35s&1.86s&19.64s&126\ 192\\
\hline
0,33&0,8495&0,79023&0,79020&1.87s&0,0027\%&45.70s&1.85s&19.48s&120\ 318\\
\hline
0,44&0,6432&0,53997&0,53989&1.88s&0,0139\%&35.63s&1.68s&13.53s&96\ 503\\
\hline
\end{tabular}
\end{center}
\caption{Resolution : $20$\ ; $N_{rom}=4$}
%
\end{figure}

\ligneinter
The same thing can be done for $50$ nodes per side of the cube :

\begin{figure}[H]%[!p]
%
\begin{center}
\begin{tabular}{|c|c|c|c|}
\hline
\includegraphics[width=0.2\linewidth, height=2.2cm]{../Figures3D/sol_1_sur8sph_un_rayres50.png}%
&%
\includegraphics[width=0.2\linewidth, height=2.2cm]{../Figures3D/sol_2_sur8sph_un_rayres50.png}%
&%
\includegraphics[width=0.2\linewidth, height=2.2cm]{../Figures3D/sol_3_sur8sph_un_rayres50.png}%
&%
\includegraphics[width=0.2\linewidth, height=2.2cm]{../Figures3D/sol_4_sur8sph_un_rayres50.png}%
\\
\hline
\includegraphics[width=0.2\linewidth, height=2.2cm]{../Figures3D/sol_5_sur8sph_un_rayres50.png}%
&%
\includegraphics[width=0.2\linewidth, height=2.2cm]{../Figures3D/sol_6_sur8sph_un_rayres50.png}%
&%
\includegraphics[width=0.2\linewidth, height=2.2cm]{../Figures3D/sol_7_sur8sph_un_rayres50.png}%
&%
\includegraphics[width=0.2\linewidth, height=2.2cm]{../Figures3D/sol_8_sur8sph_un_rayres50.png}%
\\
\hline
\end{tabular}
\end{center}
\caption{$8$ training solutions}
%
\end{figure}

\begin{figure}[H]
\begin{center}
\begin{tabular}{|c|c|c|c|}
\hline
\includegraphics[width=0.2\linewidth, height=2.2cm]{../Figures3D/snap_interpam_1_sur8sph_un_rayres50.png}%
&%
\includegraphics[width=0.2\linewidth, height=2.2cm]{../Figures3D/snap_interpam_2_sur8sph_un_rayres50.png}%
&%
\includegraphics[width=0.2\linewidth, height=2.2cm]{../Figures3D/snap_interpam_3_sur8sph_un_rayres50.png}%
&%
\includegraphics[width=0.2\linewidth, height=2.2cm]{../Figures3D/snap_interpam_4_sur8sph_un_rayres50.png}%
\\
\hline
\includegraphics[width=0.2\linewidth, height=2.2cm]{../Figures3D/snap_interpam_5_sur8sph_un_rayres50.png}%
&%
\includegraphics[width=0.2\linewidth, height=2.2cm]{../Figures3D/snap_interpam_6_sur8sph_un_rayres50.png}%
&%
\includegraphics[width=0.2\linewidth, height=2.2cm]{../Figures3D/snap_interpam_7_sur8sph_un_rayres50.png}%
&%
\includegraphics[width=0.2\linewidth, height=2.2cm]{../Figures3D/snap_interpam_8_sur8sph_un_rayres50.png}%
\\
\hline
\end{tabular}
\end{center}
\caption{Resolution : $50$\ ; extrapolated snapshots}
\end{figure}

\begin{figure}[H]
\begin{center}
\begin{tabular}{|c|c|}
\hline
\includegraphics[width=0.4\linewidth, height=3cm]{../Figures3D/ener_vp_sph_un_ray_res50.png}
&%
\includegraphics[width=0.4\linewidth, height=3cm]{../Figures3D/ener_cumul_vp_sph_un_ray_res50.png}
\\ \hline
\end{tabular}
\end{center}
\caption{Energy of POD modes : $N_{rom}^{99,99\%}=3$}
\end{figure}

\begin{figure}[H]%<+->
%
\begin{center}
\begin{tabular}{|c|c|c|}
\hline
\includegraphics[width=0.2\linewidth, height=2.2cm]{../Figures3D/phi_1_sph_un_ray_res50.png}%
&%
\includegraphics[width=0.2\linewidth, height=2.2cm]{../Figures3D/phi_2_sph_un_ray_res50.png}%
&%
\includegraphics[width=0.2\linewidth, height=2.2cm]{../Figures3D/phi_3_sph_un_ray_res50.png}%
\\
\hline
\end{tabular}
\end{center}
%
\caption{Truncated POD basis on $\Omega_{fluid}^0$ : the unit cube}
\end{figure}
%%

\begin{figure}[H]%<+->
%
\begin{center}
\begin{tabular}{|c|c|c|c|}
\hline
\includegraphics[width=0.2\linewidth, height=2.2cm]{../Figures3D/sol_rom11_sur8sph_un_rayres50.png}%
&%
\includegraphics[width=0.2\linewidth, height=2.2cm]{../Figures3D/sol_rom22_sur8sph_un_rayres50.png}%
&%
\includegraphics[width=0.2\linewidth, height=2.2cm]{../Figures3D/sol_rom33_sur8sph_un_rayres50.png}%
&%
\includegraphics[width=0.2\linewidth, height=2.2cm]{../Figures3D/sol_rom44_sur8sph_un_rayres50.png}%
\\
\hline
\end{tabular}
\end{center}
%
\caption{Solution with reduced order model on $\Omega_{fluid}^{new}$ : $50$ nodes per unit segment}
\end{figure}

\begin{figure}[H]%{Results}
%
\begin{center}
\begin{tabular}{|c|c||c|c||c|c||c|c||c||c|}
\hline
\rowcolor{lightgray} $\rho^{new}$&Porosity&${D_k^{hom,ROM}}_{11}$&${D_k^{hom,FEM}}_{11}$&Meshing&$Err$&$\phi_i^{new}$&ROM&FEM&Nodes\\
\hline
\rowcolor{yellow} 0,11&0,99442&0,99193&0,99170&20.89s&0,0242\%&586.03s&14.29s&3300.58s&1\ 910\ 451\\
\hline
\rowcolor{yellow} 0,22&0,95540&0,93465&0,93462&20.30s&0,0030\%&551.64s&13.40s&4070.04s&1\ 784\ 718\\
\hline
\rowcolor{yellow} 0,33&0,849467&0,789899&0,789897&20.41s&0,00018\%&511.61s&12.71s&1061.76s&1\ 708\ 464\\
\hline
0,44&0,64318&0,53969&0,53967&18.03s&0,0032\%&385.18s&10.81s&229.80s&1\ 313\ 223\\
\hline
\end{tabular}
\end{center}
\caption{Resolution : $50$\ ; $N_{rom}=3$}
%
\end{figure}

\ligneinter
In purpose of evaluating the ROM method, we derive from above results the ratio $\mathcal{R}_{ROM}^{99,99\%}$ :

\begin{equation}
\label{eval_rom}
\mathcal{R}_{ROM}^{99,99\%}=\dfrac{t_{\phi^{new}}+t_{ROM}+t_{\text{Meshing}}}{t_{FEM}+t_{\text{Meshing}}}
\end{equation}

Since we can see that the main contribution to $\mathcal{R}_{ROM}^{99,99\%}$ will be the time of interpolation $t_{\phi^{new}}$, %
we also evaluate the folowing contribution to $\mathcal{R}_{ROM}^{99,99\%}$ :

\begin{equation}
\label{eval_interp}
\mathcal{R}_{interpol}^{99,99\%}=\dfrac{t_{\phi^{new}}}{t_{FEM}+t_{\text{Meshing}}}
\end{equation}

The results we have found in case of a single spherical inclusion for each elementary cell, for $20$ and $50$ nodes on each side of the cube, give the folllowing tables :

\begin{figure}[H]%{Results}
%
\begin{table}[H]
\begin{center}
%
\subfloat[res20sph_unR][Resolution : $20$\ ; $N_{rom}=4$]$&$\mathcal{R}_{interpol}^{99,99\%}$&Difference&Nodes\\
\hline
0,11&2.57&2.38&0.18&136\ 002\\
\hline
0,22&2.47&2.30&0.17&126\ 192\\
\hline
0,33&2.38&2.20&0.18&120\ 318\\
\hline
0,44&2.51&2.28&0.23&96\ 503\\
\hline
\end{tabular}%
}%
\qquad
\subfloat[res50sph_unR][Resolution : $50$\ ; $N_{rom}=3$]$&$\mathcal{R}_{interpol}^{99,99\%}$&Difference&Nodes\\
\hline
\rowcolor{yellow} 0,11&0.19&0.18&0.01&1\ 910\ 451\\
\hline
\rowcolor{yellow} 0,22&0.14&0.13&0.01&1\ 784\ 718\\
\hline
\rowcolor{yellow} 0,33&0.50&0.47&0.03&1\ 708\ 464\\
\hline
0,44&1.67&1.55&0.11&1\ 313\ 223\\
\hline
\end{tabular}%
}%
%
\end{center}
\end{table}
%
\caption{Evaluation of the method : a single spherical inclusion}
\end{figure}

\etoile
\newlength{\currentparskip}
\setlength{\currentparskip}{\parskip}
\begin{minipage}{\linewidth}
\setlength{\parskip}{\currentparskip}
Sice ratio $\mathcal{R}_{ROM}^{99,99\%}$ is greater than $1$ for the smallest resolution, %
the method isn't workable for this configuration.

\par
Nevertheless in most cases, but not all of them, the method is efficient for $50$ nodes per side of the cube.

\par
We can also see that the main contribution to this excessive time of execution is the interpolation, which isn't propely a step of POD. %
It is necessary since the space in which lies the vector field $\chi(\rho,\mathbf{y})$ depends on the parameter $\rho$ of the POD : %
this is a feature of our particular problem, and we must overcome this issue in a way specific to homogenization.

\par
Notice that :

\begin{enumerate}[label=(interp \roman*)]
\item Time $t_{\phi^{new}}$ depends on the nomber of modes we keep : %
a threshold of $99,99\%$ is excessive, since FEM used in the training step of the method gives us, %
in case of $20$ nodes per side of the cube, $D_{hom}$ with a precision of $10^{-3}$ ;
%
\item Interoplation is parallelisable :
\begin{description}
\item [With respect to $i$ :] Each thread interpolates one POD mode : time of execution becomes $\max\limits_{i\in 1\dots N_{rom}}t_{\phi_i}$ ;
\item [Across each $\phi_i$ :] Interpolation of a continuous function can be done in a finite set of subdomains of the arrival space : %
time of execution is divided by the number of threads.
\end{description}
\item Although it is not fully proven mathematically, we have found a way to compute the vector field $\chi$, and the homogenized tensor $D^{hom}$ with a great precision : %
there must exist a formalism in which our problem fully lies.
\end{enumerate}
\end{minipage}

%\ligneinter
%In the following, we use the POD-interpolation method for FE spaces with a higher number of nodes, %
%then we discuss its efficiency.

In the following, we give the same results for the geometries seen in section \ref{hom_d}.

\subsubsection{A single cylinder}

\begin{figure}[H]%<+->
%
\begin{center}
\subfloat[Solution with reduced order model on $\Omega_{fluid}^{new}$]{%
%\begin{center}
\begin{tabular}{|c|c|c|c|}
\hline
\includegraphics[width=0.2\linewidth, height=2.2cm]{../Figures3D/sol_rom11_sur8cyl_un_rayres20.png}%
&%
\includegraphics[width=0.2\linewidth, height=2.2cm]{../Figures3D/sol_rom22_sur8cyl_un_rayres20.png}%
&%
\includegraphics[width=0.2\linewidth, height=2.2cm]{../Figures3D/sol_rom33_sur8cyl_un_rayres20.png}%
&%
\includegraphics[width=0.2\linewidth, height=2.2cm]{../Figures3D/sol_rom44_sur8cyl_un_rayres20.png}%
\\
\hline
\end{tabular}
%\end{center}
}%
%

%
\subfloat[Resolution : $20$\ ; $N_{rom}=5$]{%
%\begin{center}
\begin{tabular}{|c|c||c|c||c|c||c|c||c||c|}
\hline
\rowcolor{lightgray} $\rho^{new}$&Porosity&${D_k^{hom,ROM}}_{11}$&${D_k^{hom,FEM}}_{11}$&Meshing&$Err$&$\phi_i^{new}$&ROM&FEM&Nodes\\
\hline
0,11&0,96199&0,92874&0,92770&2.24s&0,1116\%&66.12s&2.10s&102.12s&132\ 963\\
\hline
0,22&0,84795&0,73657&0,73661&1.68s&0,0056\%&55.50s&1.94s&17.41s&114\ 534\\
\hline
0,33&0,65789&0,48873&0,48874&1.60s&0,0029\%&45.79s&1.83s&13.29s&97\ 029\\
\hline
0,44&0,39179&0,22272&0,22260&1.38s&0,0523\%&30.74s&1.58s&8.62s&69\ 612\\
\hline
\end{tabular}
%\end{center}
%\caption{Resolution : $20$\ ; $N_{rom}=5$}
%
}
\end{center}
\caption{Results of ROM with interpolation}
\end{figure}

\begin{figure}[H]%<+->
%
\begin{center}
%
\subfloat[Solution with reduced order model on $\Omega_{fluid}^{new}$]{%
\begin{tabular}{|c|c|c|c|}
\hline
\includegraphics[width=0.2\linewidth, height=2.2cm]{../Figures3D/sol_rom11_sur8cyl_un_rayres50.png}%
&%
\includegraphics[width=0.2\linewidth, height=2.2cm]{../Figures3D/sol_rom22_sur8cyl_un_rayres50.png}%
&%
\includegraphics[width=0.2\linewidth, height=2.2cm]{../Figures3D/sol_rom33_sur8cyl_un_rayres50.png}%
&%
\includegraphics[width=0.2\linewidth, height=2.2cm]{../Figures3D/sol_rom44_sur8cyl_un_rayres50.png}%
\\
\hline
\end{tabular}
%
%\caption{Solution with reduced order model on $\Omega_{fluid}^{new}$ : $50$ nodes per unit segment}
}%
%

\subfloat[Resolution : $50$\ ; $N_{rom}=5$]{%
\begin{tabular}{|c|c||c|c||c|c||c|c||c||c|}
\hline
\rowcolor{lightgray} $\rho^{new}$&Porosity&${D_k^{hom,ROM}}_{11}$&${D_k^{hom,FEM}}_{11}$&Meshing&$Err$&$\phi_i^{new}$&ROM&FEM&Nodes\\
\hline
\rowcolor{yellow} 0,11&0,96199&0,92734&0,92691&20.51s&0,0464\%&959.15s&19.52s&1239.23s&1\ 892\ 742\\
\hline
0,22&0,84795&0,736090&0,736086&18.80s&0,0005\%&829.28s&17.22s&801.12s&1\ 644\ 420\\
\hline
0,33&0,65788&0,48861&0,48860&15.96s&0,0003\%&662.10s&14.28s&222.02s&1\ 331\ 142\\
\hline
0,44&0,39179&0,22267&0,22265&11.73s&0,0091\%&396.81s&9.23s&0.13s&833\ 502\\
\hline
\end{tabular}
}
%
\end{center}
\caption{Results of ROM with interpolation}
%
\end{figure}

\ligneinter

\begin{figure}[H]%{Results}
%
\begin{table}[H]
\begin{center}
%
\subfloat[res20cyl_unR][Resolution : $20$\ ; $N_{rom}=5$]$&$\mathcal{R}_{interpol}^{99,99\%}$&Difference&Nodes\\
\hline
0,11&0.68&0.63&0.04&132\ 963\\
\hline
0,22&3.10&2.91&0.19&114\ 534\\
\hline
0,33&3.31&3.07&0.23&97\ 029\\
\hline
0,44&3.37&3.07&0.30&69\ 612\\
\hline
\end{tabular}%
}%
\qquad
\subfloat[res50cyl_unR][Resolution : $50$\ ; $N_{rom}=5$]$&$\mathcal{R}_{interpol}^{99,99\%}$&Difference&Nodes\\
\hline
\rowcolor{yellow} 0,11&0.79&0.76&0.03&1\ 892\ 742\\
\hline
0,22&1.06&1.01&0.04&1\ 644\ 420\\
\hline
0,33&2.91&2.61&0.13&1\ 331\ 142\\
\hline
0,44&2.62&2.49&0.13&833\ 502\\
\hline
\end{tabular}%
}%
%
\end{center}
\end{table}
%
\caption{Evaluation of the method : a single cylinder inclusion}
\end{figure}

%\Floatbarrier
\subsubsection{Two spheres}

\begin{figure}[H]%<+->
%
\begin{center}
%
\subfloat[Solution with reduced order model on $\Omega_{fluid}^{new}$]{%
\begin{tabular}{|c|c|c|c|}
\hline
\includegraphics[width=0.2\linewidth, height=2.2cm]{../Figures3D/sol_rom11_sur82sph_rayres20.png}%
&%
\includegraphics[width=0.2\linewidth, height=2.2cm]{../Figures3D/sol_rom22_sur82sph_rayres20.png}%
&%
\includegraphics[width=0.2\linewidth, height=2.2cm]{../Figures3D/sol_rom33_sur82sph_rayres20.png}%
&%
\includegraphics[width=0.2\linewidth, height=2.2cm]{../Figures3D/sol_rom44_sur82sph_rayres20.png}%
\\
\hline
\end{tabular}
}

\subfloat[Resolution : $20$\ ; $N_{rom}=3$]{%
\begin{tabular}{|c|c||c|c||c|c||c|c||c||c|}
\hline
\rowcolor{lightgray} $\rho^{new}$&Porosity&${D_k^{hom,ROM}}_{11}$&${D_k^{hom,FEM}}_{11}$&Meshing&$Err$&$\phi_i^{new}$&ROM&FEM&Nodes\\
\hline
0,11&0,98029&0,97370&0,97367&2.36s&0,0024\%&43.00s&2.18s&21.18s&143\ 322\\
\hline
0,22&0,94126&0,93500&0,93503&1.88s&0,0033\%&40.09s&2.11s&21.38s&134\ 229\\
\hline
0,33&0,83533&0,83002&0,83009&1.94s&0,0093\%&36.27s&2.03s&17.50s&125\ 094\\
\hline
0,44&0,62904&0,62589&0,62591&1.95s&0,0036\%&28.85s&1.84s&15.55s&100\ 767\\
\hline
\end{tabular}
}
%
%\caption{Resolution : $20$\ ; $N_{rom}=3$}
%
\end{center}
%
\caption{Results of ROM with interpolation}
\end{figure}

\begin{figure}[H]
%
\begin{center}
%
\subfloat[Solution with reduced order model on $\Omega_{fluid}^{new}$]{%
\begin{tabular}{|c|c|c|c|}
\hline
\includegraphics[width=0.2\linewidth, height=2.2cm]{../Figures3D/sol_rom11_sur82sph_rayres50.png}%
&%
\includegraphics[width=0.2\linewidth, height=2.2cm]{../Figures3D/sol_rom22_sur82sph_rayres50.png}%
&%
\includegraphics[width=0.2\linewidth, height=2.2cm]{../Figures3D/sol_rom33_sur82sph_rayres50.png}%
&%
\includegraphics[width=0.2\linewidth, height=2.2cm]{../Figures3D/sol_rom44_sur82sph_rayres50.png}%
\\
\hline
\end{tabular}
}
%

\subfloat[Resolution : $50$\ ; $N_{rom}=2$]{%
\begin{tabular}{|c|c||c|c||c|c||c|c||c||c|}
\hline
\rowcolor{lightgray} $\rho^{new}$&Porosity&${D_k^{hom,ROM}}_{11}$&${D_k^{hom,FEM}}_{11}$&Meshing&$Err$&$\phi_i^{new}$&ROM&FEM&Nodes\\
\hline
\rowcolor{yellow} 0,11&0,98029&0,97353&0,97347&20.40s&0,0064\%&373.63s&18.42s&951.97s&1\ 841\ 469\\
\hline
\rowcolor{yellow} 0,22&0,94126&0,93479&0,94394&20.88s&0,0048\%&353.76s&17.89s&1098.30s&1\ 736\ 505\\
\hline
\rowcolor{yellow} 0,33&0,83533&0,82986&0,82994&20.40s&0,0091\%&335.76s&18.44s&803.21s&1\ 671\ 780\\
\hline
0,44&0,62904&0,62574&0,62582&17.73s&0,0131\%&256.65s&14.12s&224.55s&1\ 294\ 941\\
\hline
\end{tabular}
}
%
\end{center}
\caption{Results of ROM with interpolation}
%
\end{figure}

\ligneinter
\begin{figure}[H]%{Results}
%
\begin{table}[H]
\begin{center}
%
\subfloat[res202sphR][Resolution : $20$\ ; $N_{rom}=3$]$&$\mathcal{R}_{interpol}^{99,99\%}$&Difference&Nodes\\
\hline
0,11&1.99&1.80&0.19&143\ 322\\
\hline
0,22&1.90&1.73&0.17&134\ 229\\
\hline
0,33&2.07&1.87&0.20&125\ 094\\
\hline
0,44&1.87&1.65&0.22&100\ 767\\
\hline
\end{tabular}%
}%
\qquad
\subfloat[res502sphR][Resolution : $50$\ ; $N_{rom}=2$]$&$\mathcal{R}_{interpol}^{99,99\%}$&Difference&Nodes\\
\hline
\rowcolor{yellow} 0,11&0.42&0.38&0.04&1\ 841\ 469\\
\hline
\rowcolor{yellow} 0,22&0.35&0.32&0.03&1\ 736\ 505\\
\hline
\rowcolor{yellow} 0,33&0.45&0.40&0.05&1\ 671\ 780\\
\hline
0,44&1.19&1.06&0.13&1\ 294\ 941\\
\hline
\end{tabular}%
}%
%
\end{center}
\end{table}
%
\caption{Evaluation of the method : two spheres}
\end{figure}

\subsubsection{One sphere and one cylinder : $\rho$ is the radius of the sphere}

\begin{figure}[H]%<+->
%
\begin{center}
%
\subfloat[Solution with reduced order model on $\Omega_{fluid}^{new}$]{%
\begin{tabular}{|c|c|c|c|}
\hline
\includegraphics[width=0.2\linewidth, height=2.2cm]{../Figures3D/sol_rom11_sur8cylsph_ray_sphres20.png}%
&%
\includegraphics[width=0.2\linewidth, height=2.2cm]{../Figures3D/sol_rom22_sur8cylsph_ray_sphres20.png}%
&%
\includegraphics[width=0.2\linewidth, height=2.2cm]{../Figures3D/sol_rom33_sur8cylsph_ray_sphres20.png}%
&%
\includegraphics[width=0.2\linewidth, height=2.2cm]{../Figures3D/sol_rom44_sur8cylsph_ray_sphres20.png}%
\\
\hline
\end{tabular}
}
%

%
\subfloat[Resolution : $20$\ ; $N_{rom}=3$]{%
\begin{tabular}{|c|c||c|c||c|c||c|c||c||c|}
\hline
\rowcolor{lightgray} $\rho^{new}$&Porosity&${D_k^{hom,ROM}}_{11}$&${D_k^{hom,FEM}}_{11}$&Meshing&$Err$&$\phi_i^{new}$&ROM&FEM&Nodes\\
\hline
0,11&0,98029&0,863876&0,86358&2.28s&0,0204\%&40.23s&2.06s&19.16s&132\ 900\\
\hline
0,22&0,94126&0,82761&0,82797&1.79s&0,0433\%&36.09s&2.01s&18.26s&121\ 956\\
\hline
0,33&0,0.77878&0,73056&0,73060&1.95s&0,0047\%&33.47s&1.93s&15.24s&116\ 355\\
\hline
0,44&0,57250&0,54089&0,540046&1.89s&0,1554\%&25.41s&1.77s&13.28s&92\ 058\\
\hline
\end{tabular}
}
%
\end{center}
%
\caption{Results of ROM with interpolation}
\end{figure}

\begin{figure}[H]%<+->
%
\begin{center}
%
\subfloat[Solution with reduced order model on $\Omega_{fluid}^{new}$]{%
\begin{tabular}{|c|c|c|c|}
\hline
\includegraphics[width=0.2\linewidth, height=2.2cm]{../Figures3D/sol_rom11_sur8cylsph_ray_sphres50.png}%
&%
\includegraphics[width=0.2\linewidth, height=2.2cm]{../Figures3D/sol_rom22_sur8cylsph_ray_sphres50.png}%
&%
\includegraphics[width=0.2\linewidth, height=2.2cm]{../Figures3D/sol_rom33_sur8cylsph_ray_sphres50.png}%
&%
\includegraphics[width=0.2\linewidth, height=2.2cm]{../Figures3D/sol_rom44_sur8cylsph_ray_sphres50.png}%
\\
\hline
\end{tabular}
}
%

\subfloat[Resolution : $50$\ ; $N_{rom}=2$]{%
\begin{tabular}{|c|c||c|c||c|c||c|c||c||c|}
\hline
\rowcolor{lightgray} $\rho^{new}$&Porosity&${D_k^{hom,ROM}}_{11}$&${D_k^{hom,FEM}}_{11}$&Meshing&$Err$&$\phi_i^{new}$&ROM&FEM&Nodes\\
\hline
\rowcolor{yellow} 0,11&0,92374&0,86362&0,86301&18.54s&0,0709\%&341.82s&17.17s&930.87s&1\ 673\ 688\\
\hline
\rowcolor{yellow} 0,22&0,88471&0,82716&0,82747&18.79s&0,0369\%&325.36s&15.93s&791.92s&1\ 630\ 377\\
\hline
\rowcolor{yellow} 0,33&0,77878&0,729139&0,730195&18.47s&0,1446\%&300.91s&15.21s&1722.75s&1\ 518\ 327\\
\hline
0,44&0,57250&0,542528&0,539777&16.6s&0,5096\%&229.21s&12.2s&143.36s&1\ 167\ 783\\
\hline
\end{tabular}
}
%
\end{center}
%
\caption{Results of ROM with interpolation}
\end{figure}

\ligneinter

\begin{figure}[H]%{Results}
%
\begin{table}[H]
\begin{center}
%
\subfloat[res20cylsph_rsR][Resolution : $20$\ ; $N_{rom}=3$]$&$\mathcal{R}_{interpol}^{99,99\%}$&Difference&Nodes\\
\hline
0,11&2.08&1.88&0.20&132\ 900\\
\hline
0,22&1.99&1.80&0.18&121\ 956\\
\hline
0,33&2.13&1.91&0.22&116\ 355\\
\hline
0,44&1.92&1.67&0.24&92\ 058\\
\hline
\end{tabular}%
}%
\qquad
\subfloat[res50cylsph_rsR][Resolution : $50$\ ; $N_{rom}=2$]$&$\mathcal{R}_{interpol}^{99,99\%}$&Difference&Nodes\\
\hline
\rowcolor{yellow} 0,11&0.40&0.36&0.04&1\ 673\ 688\\
\hline
\rowcolor{yellow} 0,22&0.44&0.40&0.04&1\ 630\ 377\\
\hline
\rowcolor{yellow} 0,33&0.19&0.17&0.02&1\ 518\ 327\\
\hline
0,44&1.61&1.43&0.18&1\ 167\ 783\\
\hline
\end{tabular}%
}%
%
\end{center}
\end{table}
%
\caption{Evaluation of the method : one sphere and one cylinder, $\rho$ is the radius of the sphere}
\end{figure}

\subsubsection{One sphere and one cylinder : $\rho$ is the radius of the cylinder}

\begin{figure}[H]%<+->
%
\begin{center}
%
\subfloat[Solution with reduced order model on $\Omega_{fluid}^{new}$]{%
\begin{tabular}{|c|c|c|c|}
\hline
\includegraphics[width=0.2\linewidth, height=2.2cm]{../Figures3D/sol_rom11_sur8cylsph_ray_cylres20.png}%
&%
\includegraphics[width=0.2\linewidth, height=2.2cm]{../Figures3D/sol_rom22_sur8cylsph_ray_cylres20.png}%
&%
\includegraphics[width=0.2\linewidth, height=2.2cm]{../Figures3D/sol_rom33_sur8cylsph_ray_cylres20.png}%
&%
\includegraphics[width=0.2\linewidth, height=2.2cm]{../Figures3D/sol_rom44_sur8cylsph_ray_cylres20.png}%
\\
\hline
\end{tabular}
}
%

\subfloat[Resolution : $20$\ ; $N_{rom}=4$]{%
\begin{tabular}{|c|c||c|c||c|c||c|c||c||c|}
\hline
\rowcolor{lightgray} $\rho^{new}$&Porosity&${D_k^{hom,ROM}}_{11}$&${D_k^{hom,FEM}}_{11}$&Meshing&$Err$&$\phi_i^{new}$&ROM&FEM&Nodes\\
\hline
0,11&0,94785&0,91523&0,91425&2.78s&0,1068\%&54.51s&2.25s&19.86s&138\ 420\\
\hline
0,22&0,83381&0,72475&0,72472&1.74s&0,0041\%&47.22s&2.23s&17.76s&120\ 540\\
\hline
0,33&0,64374&0,47863&0,47869&1.58s&0,0085\%&36.70s&1.75s&12.83s&98\ 262\\
\hline
0,44&0,37765&0,21372&0,21368&1.35s&0,0193\%&21.90s&1.48s&6.84s&62\ 127\\
\hline
\end{tabular}
}
%
\end{center}
%
\caption{Results of ROM with interpolation}%
\end{figure}

\begin{figure}[H]%<+->
%
\begin{center}
%
\subfloat[Solution with reduced order model on $\Omega_{fluid}^{new}$]{%
\begin{tabular}{|c|c|c|c|}
\hline
\includegraphics[width=0.2\linewidth, height=2.2cm]{../Figures3D/sol_rom11_sur8cylsph_ray_cylres50.png}%
&%
\includegraphics[width=0.2\linewidth, height=2.2cm]{../Figures3D/sol_rom22_sur8cylsph_ray_cylres50.png}%
&%
\includegraphics[width=0.2\linewidth, height=2.2cm]{../Figures3D/sol_rom33_sur8cylsph_ray_cylres50.png}%
&%
\includegraphics[width=0.2\linewidth, height=2.2cm]{../Figures3D/sol_rom44_sur8cylsph_ray_cylres50.png}%
\\
\hline
\end{tabular}
}
%

%
\subfloat[Resolution : $50$\ ; $N_{rom}=4$]{%
\begin{tabular}{|c|c||c|c||c|c||c|c||c||c|}
\hline
\rowcolor{lightgray} $\rho^{new}$&Porosity&${D_k^{hom,ROM}}_{11}$&${D_k^{hom,FEM}}_{11}$&Meshing&$Err$&$\phi_i^{new}$&ROM&FEM&Nodes\\
\hline
\rowcolor{yellow} 0,11&0,94785&0,91461&0,91351&19.78s&0,1207\%&729.03s&21.66s&3191.41s&1\ 813\ 602\\
\hline
\rowcolor{yellow} 0,22&0,83381&0,72436&0,72435&18.08s&0,0016\%&646.01s&19.46s&801.00s&1\ 605\ 630\\
\hline
0,33&0,64374&0,47860&0,47859&14.16s&0,0004\%&487.25s&14.78s&240.91s&1\ 200\ 282\\
\hline
0,44&0,37765&0,21391&0,21388&9.86s&0,0128\%&293.35s&9.21s&132.92s&754\ 662\\
\hline
\end{tabular}
}
%
\end{center}
%
\caption{Results of ROM with interpolation}
\end{figure}

\ligneinter

\begin{figure}[H]%{Results}
%
\begin{table}[H]
\begin{center}
%
\subfloat[res20cylsph_rcR][Resolution : $20$\ ; $N_{rom}=4$]$&$\mathcal{R}_{interpol}^{99,99\%}$&Difference&Nodes\\
\hline
0,11&2.63&2.42&0.20&120\ 540\\
\hline
0,22&2.63&2.42&0.20&120\ 240\\
\hline
0,33&2.78&2.55&0.23&98\ 262\\
\hline
0,44&3.02&2.67&0.35&62\ 127\\
\hline
\end{tabular}%
}%
\qquad
\subfloat[res50cylsph_rcR][Resolution : $50$\ ; $N_{rom}=4$]$&$\mathcal{R}_{interpol}^{99,99\%}$&Difference&Nodes\\
\hline
\rowcolor{yellow} 0,11&0.24&0.23&0.01&1\ 813\ 602\\
\hline
\rowcolor{yellow} 0,22&0.83&0.79&0.05&1\ 605\ 630\\
\hline
0,33&2.02&1.91&0.11&1\ 200\ 282\\
\hline
0,44&2.19&2.06&0.13&754\ 662\\
\hline
\end{tabular}%
}%
%
\end{center}
\end{table}
%
\caption{Evaluation of the method : one sphere and one cylinder, $\rho$ is the radius of the cylinder}
\end{figure}
%%%%%%%%%%%%%%%%%%%%%%%%%%%%%%%%%%%%%%%%%%%%%%%%%%%%%%%%%%%%%%%%%%%%%%%%%%%%%%%%%%%%%%%%%%%%%%%%%%%%%%%%%%%%%%%%%%%%%%%%%%%%%%%%%%%%%%%%%%%%%%%%%%%%%%
%%%%%%%%%%%%%%%%%%%%%%%%%%%%%%%%%%%%%%%%%%%%%%%%%%%%%%%%%%%%%%%%%%%%%%%%%%%%%%%%%%%%%%%%%%%%%%%%%%%%%%%%%%%%%%%%%%%%%%%%%%%%%%%%%%%%%%%%%%%%%%%%%%%%%%

\subsection{Discussion}









