\section{Results and discussion}\label{res}
%\subsection{Numerical simulations}

%CPU $\times$ 8 \dots

\subsection{Material}

\begin{description}
\item [Machine] 1$\times$CPU 2.90GHz
\item [FEM] Resolution of the weak problem in $\chi$ was done with FEniCS, %
an open-source platform programmed in Python designed for Finite Element Method.
\item [ROM] The numpy package of python 3 was used for POD.
\end{description}

\subsection{Two-dimensional case : varying the radius of a single circular inclusion}

\dotfill

\subsection{Three-dimensional case}

\subsubsection{A single spherical inclusion}

\begin{figure}[H]
%
\begin{center}
\begin{tabular}{|c|c|c|c|}
\hline
\includegraphics[width=0.2\linewidth, height=2.2cm]{../Figures3D/sol_1_sur8sph_un_ray.png}%
&%
\includegraphics[width=0.2\linewidth, height=2.2cm]{../Figures3D/sol_2_sur8sph_un_ray.png}%
&%
\includegraphics[width=0.2\linewidth, height=2.2cm]{../Figures3D/sol_3_sur8sph_un_ray.png}%
&%
\includegraphics[width=0.2\linewidth, height=2.2cm]{../Figures3D/sol_4_sur8sph_un_ray.png}%
\\
\hline
\includegraphics[width=0.2\linewidth, height=2.2cm]{../Figures3D/sol_5_sur8sph_un_ray.png}%
&%
\includegraphics[width=0.2\linewidth, height=2.2cm]{../Figures3D/sol_6_sur8sph_un_ray.png}%
&%
\includegraphics[width=0.2\linewidth, height=2.2cm]{../Figures3D/sol_7_sur8sph_un_ray.png}%
&%
\includegraphics[width=0.2\linewidth, height=2.2cm]{../Figures3D/sol_8_sur8sph_un_ray.png}%
\\
\hline
\end{tabular}
\end{center}
\caption{$8$ solutions of the physical problem}
%
\end{figure}

%\begin{itemize}
%\item $\rho^j$ is the radius of a sphere at the center of the cell.
%\item We interpolate the $\chi(\mathbf{y},\rho^j)$ on the unit cube minus $\mathcal{S}(\mathbf{0},0,0001)$.
%\end{itemize}

%\etoile

\begin{figure}[H]%<+->
%
\begin{center}
\begin{tabular}{|c|c|c|c|}
\hline
\includegraphics[width=0.2\linewidth, height=2.2cm]{../Figures3D/phi_1_sph_un_ray.png}%
&%
\includegraphics[width=0.2\linewidth, height=2.2cm]{../Figures3D/phi_2_sph_un_ray.png}%
&%
\includegraphics[width=0.2\linewidth, height=2.2cm]{../Figures3D/phi_3_sph_un_ray.png}%
&%
\includegraphics[width=0.2\linewidth, height=2.2cm]{../Figures3D/phi_4_sph_un_ray.png}%
\\
\hline
\includegraphics[width=0.2\linewidth, height=2.2cm]{../Figures3D/phi_5_sph_un_ray.png}%
&%
\includegraphics[width=0.2\linewidth, height=2.2cm]{../Figures3D/phi_6_sph_un_ray.png}%
&%
\includegraphics[width=0.2\linewidth, height=2.2cm]{../Figures3D/phi_7_sph_un_ray.png}%
&%
\includegraphics[width=0.2\linewidth, height=2.2cm]{../Figures3D/phi_8_sph_un_ray.png}%
\\
\hline
\end{tabular}
\end{center}
%
\caption{The POD basis on $\Omega_{fluid}^0$ : the unit cube minus $\mathcal{S}(\mathbf{0};0,0001)$}
\end{figure}

\begin{figure}[H]
\begin{center}
\begin{tabular}{|c|c|}
\hline
\includegraphics[width=0.4\linewidth, height=3cm]{../Figures3D/ener_vp_sph_un_ray.png}
&%
\includegraphics[width=0.4\linewidth, height=3cm]{../Figures3D/ener_cumul_vp_sph_un_ray.png}
\\ \hline
\end{tabular}
\end{center}
\caption{Energy of POD modes}
\end{figure}

\begin{multicols}{2}
Cumulated energy :

\columnbreak
\begin{itemize}
\item $99,99\%$ : $N_{rom}=4$
\item $99,9\%$ : $N_{rom}=2$
\end{itemize}
%
\end{multicols}

\ligneinter

\begin{figure}[H]%{Results}
%
\begin{center}
\begin{tabular}{|c|c||c|c||c|c||c|c||c||c|}
\hline
$\rho^{new}$&Porosity&${D_k^{hom,ROM}}_{11}$&${D_k^{hom,FEM}}_{11}$&$Err$&$\phi_i^{new}$&ROM&FEM&Nodes\\%&$Err$&$\phi_i^{new}$&ROM&FEM\\
\hline
0,22&0,9553&0,9357&0,9356&0,011\%&10s&$<$1s&75s&20724\\
\hline
0,33&0,8494&0,7911&0,7910&0,013\%&9s&$<$1s&60s&20544\\
\hline
0,44&0,6431&0,5406&0,5405&0,018\%&6s&$<$1s&30s&16953\\
\hline
\end{tabular}
\end{center}
\caption{$\Omega_f^0=\dots$, $N_{rom}=4$}
%
\end{figure}
%%

\etoile

%%
\begin{figure}[H]%{Results}
%
\begin{center}
\begin{tabular}{|c|c||c|c||c|c||c|c||c||c|}
\hline
$\rho^{new}$&Porosity&${D_k^{hom,ROM}}_{11}$&${D_k^{hom,FEM}}_{11}$&$Err$&$\phi_i^{new}$&ROM&FEM&Nodes\\%&$Err$&$\phi_i^{new}$&ROM&FEM\\
\hline
0,22&0,9553&\textbf{0.9363}&0,9356&\textbf{0.086}\%&\textbf{5}s&$<$1s&75s&20724\\
\hline
0,33&0,8494&\textbf{0.7992}&0,7910&\textbf{0.013}\%&\textbf{4}s&$<$1s&60s&20544\\
\hline
0,44&0,6431&\textbf{0.5408}&0,5405&\textbf{0.048}\%&\textbf{4}s&$<$1s&30s&16953\\
\hline
\end{tabular}
\end{center}
\caption{$\Omega_f^0=\dots$, $N_{rom}=2$}
%
\end{figure}


